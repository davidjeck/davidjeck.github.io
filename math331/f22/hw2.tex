\documentclass[12pt]{article}
\usepackage[utf8]{inputenc}
\usepackage[letterpaper]{geometry}
\usepackage{amsmath}
\usepackage{amsfonts}
\usepackage{mathrsfs}
\usepackage{amsthm}
\usepackage{amssymb}
\usepackage{color}
\setlength{\topmargin}{-0.4 true in}
\setlength{\headsep}{0 true in}
\setlength{\topskip}{0 true in}
\setlength{\textwidth}{6.5 true in}
\setlength{\oddsidemargin}{0 true in}
\setlength{\evensidemargin}{0 true in}
\setlength{\textheight}{9.5 true in}
\renewcommand{\arraystretch}{1.2}
\newcommand{\eps}{\varepsilon}
\newcommand{\R}{{\mathbb R}}
\newcommand{\N}{{\mathbb N}}
\newcommand{\Q}{{\mathbb Q}}
\newcommand{\Z}{{\mathbb Z}}
\theoremstyle{definition}
\newtheorem{problem}{Problem}
\definecolor{darkblue}{RGB}{0,0,150}
\definecolor{darkred}{RGB}{180,0,0}
\definecolor{darkgreen}{RGB}{0,120,0}
\newcounter{partcounter}
\newcommand{\dspparts}[1]{\medskip
   \halign{\hskip40pt\stepcounter{partcounter}\llap{\bfseries\alph{partcounter})\ }$\displaystyle ##$\qquad\hfil&&
            \hskip30pt\stepcounter{partcounter}\llap{\bfseries\alph{partcounter})\ }$\displaystyle ##$\qquad\hfil\cr
   #1\crcr}}
\newenvironment{answer}{\par\medskip\bgroup\color{darkblue}}{\egroup\par\medskip}
\newenvironment{grading}{\par\medskip\bgroup\color{darkred}}{\egroup\par\bigskip}
\newenvironment{secondans}{\bgroup\color{darkgreen}}{\egroup}
\newcommand{\grd}[1]{{\color{darkred}#1}}
\pagestyle{empty}
\allowdisplaybreaks


\begin{document}

\hbox to \hsize{\textbf{Math 331\hfill Homework 2}}
\nointerlineskip
\vskip 2pt
\hrule height 1pt

\medskip

\centerline{\textit{This homework is due by the end of the day on Friday, September 7}}
\centerline{\textit{Problems are mostly on Sections 1.3 and 1.4, not including Bolzano-Weirstrass.}}

\bigskip

\def\F{{\mathbb F}}
\def\O{{\mathcal O}}

\begin{problem}
Prove using only the defintion of real numbers as Dedekind cuts and the definitions
of $+$ and $<$ in terms of Dedekind cuts:  If $\alpha,\beta,\delta\in\R$ and
$\alpha < \beta$, then $\alpha+\delta < \beta+\delta$.
\end{problem}

\begin{answer}
% Erase this line and put your answer here!  (Problem 1, 2 points)
\end{answer}


\begin{problem}
[From Problem 1.3.7 in the textbook]\ \ 
Suppose that $(\F,+,\cdot)$ is a field, and $S\subseteq \F$. We say that $S$ is
a subfield of $\F$ if it is a field under the same addition and multiplication as $\F$.
To show that $S$ is a subfield of $\F$, it is enough to show that $0\in S$, $1\in S$,
and $S$ is closed under addition, multiplication, taking additive inverses, and
taking multiplicative inverses..

Let $\Q[\sqrt 2] = \{r+s\sqrt 2\,|\, r,s\in\Q\}$.  Show that $\Q[\sqrt 2]$ is
a subfield of $\R$.    (Note: Remember that $r$ and $s$ can be zero in $r+s\sqrt 2$.)
\end{problem}

\begin{answer}
% Erase this line and put your answer here!  (Problem 2, 5 points)
\end{answer}


\begin{problem}
[Problem 1.3.11 from the textbook]\ \ Let $(F,+,\cdot)$ be an ordered field.  Use the
definition of $x<y$ and the order axioms to prove the
transitive property of $<$.  That is, show that for any $a,b,c\in\F$, if $a<b$ and $b<c$,
then $a<c$. [Note: Since $\F$ is not necessarily $\R$, you can't use common facts that you
know about $\R$.  You can only use the actual definition and axioms.]
\end{problem}

\begin{answer}
% Erase this line and put your answer here!  (Problem 3, 2 points)
\end{answer}



\begin{problem}
\textbf{(a)} Let $\O_1,\O_2\dots,\O_k$ be some finite number of open subsets of
$\R$.  Prove that their intersection, $\bigcap_{i=1}^k\O_i$, is open.  (Hint: Use the characterization
of open that involves $\eps>0$.  Start by taking arbitrary $x\in\bigcap_{i=1}^k\O_i$.)

\textbf{(b)} Show that the intersection of an infinite number of open sets is not necessarily open
by finding $\bigcap_{n=1}^\infty\big(-1-\frac{1}{n},1+\frac{1}{n}\big)$. (Justify your answer!)
\end{problem}
\begin{answer}
% Erase this line and put your answer here! (Problem 4, 4 points)
\end{answer}


\begin{problem}
Consider the \textbf{unbounded} closed interval $[0,\infty)$.  Find an open cover of this
interval that has no finite subcover.  (This problem shows that the hypothesis that the interval is 
bounded cannot be removed from the Heine-Borel Theorem. Use a simple example, but justify your answer!)
\end{problem}

\begin{answer}
% Erase this line and put your answer here! (Problem 5, 2 points)
\end{answer}


\begin{problem}
[Problem 1.4.3 from the textbook]\ \ Suppose that $\{\O_\alpha\,|\,\alpha\in A\}$ is an open cover
of the interval $[0,1)$.  Suppose furthermore that $1\in\bigcup_{\alpha\in A}\O_\alpha$.
Prove that there is finite subcover of $[0,1)$ from $\{\O_\alpha\,|\,\alpha\in A\}$.
[This question tests your understanding of the proof of the Heine-Borel Theorem.]
\end{problem}

\begin{answer}
% Erase this line and put your answer here!  (Problem 6, 3 points)
\end{answer}


\begin{problem}
Let $f(x)$ be a real-valued function that is defined on an interval $I$.  We say that $f$ is
bounded above on $I$ if there is a number $M$ such that $f(x)<M$ for all $x\in I$.

Suppose that $f(x)$ is defined on the bounded, closed interval $[a,b]$.  Suppose that for
every $x\in[a,b]$, there is an $\eps>0$ such that $f$ is bounded above on the
interval $(x-\eps,x+\eps)$.  Use the Heine-Borel theorem to prove that
$f$ is bounded above on $[a,b]$.  (Hint: Compare this to an example about functions 
that was done in class.)
\end{problem}

\begin{answer}
% Erase this line and put your answer here! (Problem 7, 4 points)
\end{answer}


\end{document}
