\documentclass[12pt]{article}
\usepackage[utf8]{inputenc}
\usepackage[letterpaper]{geometry}
\usepackage{amsmath}
\usepackage{amsfonts}
\usepackage{mathrsfs}
\usepackage{amsthm}
\usepackage{amssymb}
\usepackage{color}
\setlength{\topmargin}{-0.2 true in}
\setlength{\headsep}{0 true in}
\setlength{\topskip}{0 true in}
\setlength{\textwidth}{6.5 true in}
\setlength{\oddsidemargin}{0 true in}
\setlength{\evensidemargin}{0 true in}
\setlength{\textheight}{8.8 true in}
\renewcommand{\arraystretch}{1.2}
\newcommand{\eps}{\varepsilon}
\newcommand{\R}{{\mathbb R}}
\newcommand{\N}{{\mathbb N}}
\newcommand{\Q}{{\mathbb Q}}
\newcommand{\Z}{{\mathbb Z}}
\theoremstyle{definition}
\newtheorem{problem}{Problem}
\definecolor{darkblue}{RGB}{0,0,150}
\definecolor{darkred}{RGB}{180,0,0}
\newenvironment{answer}{\par\medskip\bgroup\color{darkblue}}{\egroup\par\medskip}
\newenvironment{grading}{\par\medskip\bgroup\color{darkred}}{\egroup\par\bigskip}
\pagestyle{empty}


\begin{document}

\hbox to \hsize{\textbf{Math 331, Fall 2022\hfill Take-home Test \#1}}
\nointerlineskip
\vskip 3pt
\hrule height 1pt

\medskip

{\narrower\narrower\small{
This test is due by the start of class on Monday, October 3.  
\textbf{You should do any six out of the seven problems on the test.  Do not do
all seven; if you do all seven, your answer for problem~7 will be ignored.}\par
You should do this test on your own, using only your textbook, class notes, 
and previous work in the course as reference.  You should
not work with other students, you should not use the Internet or other references, and you should not
consult anyone except the professor for the course.  You can ask questions about the test in class and by email.
I might give some hints and clarifications, but I am unlikely to give
extensive, detailed help.   Be sure to show all of
your work!  If you are not able to complete a problem, you should turn in work showing
whatever progress you have made on it, for partial credit.
\textbf{Note: There will be no ``rewrites'' for this test.}\par
You can submit your answers in LaTeX on overleaf.com, if you want to do 
that, but you are welcome to write up your answers neatly by hand.\par
Although some problems are more difficult than others, and some are fairly easy, 
all six problems will count equally.}
\par}

\medskip
\hrule height 1pt
\bigskip


\def\ds{\displaystyle}


\begin{problem} % problem 1
State a careful definition of $\ds\lim_{x\to a^+}f(x)=+\infty$.  Then use the definition to
prove directly that $\ds\lim_{x\to 0^+}\frac{1}{x} = +\infty$.
\end{problem}

\begin{answer}
%You can put your answer here.
\end{answer}





\begin{problem}  % problem 2
Let $X$ and $Y$ be non-empty, bounded subsets of $\R$.  Suppose that for every $x\in X$ and 
for every $y\in Y$, $x<y$.  Prove that $lub(X)\le glb(Y)$.  Is it always true that
$lub(X)<glb(Y)\;$? (Prove or give a counterexample!)
\end{problem}

\begin{answer}
%You can put your answer here.
\end{answer}





\begin{problem}  % problem 3
Let $A$ and $B$ be subsets of $\R$.  Suppose that $x$ is an accumulation point of the set
$A\cup B$.  Show that $x$ is an accumulation point of $A$ or $x$ is an accumulation
point of $B$ (or both).   (Hint: Try a proof by contradiction.)
\end{problem}

\begin{answer}
%You can put your answer here.
\end{answer}





\begin{problem}  % problem 4
Let $f$ and $g$ be functions. Then we can define a new function $\max(f,g)$ whose value
at $x$ is given by $\max(f(x),g(x))$.
\begin{itemize}
\item[(a)] Show that for any numbers $a$ and $b$, $\max(a,b)=\frac{1}{2}\big(|a-b|+a+b\big)$.
(Hint: Consider two cases.)
\item[(b)] Now, suppose that $f$ and $g$ are continuous on an interval $I$.  Show that
the function $\max(f,g)$ is also continuous on $I$.  Be clear about what continuity rules
or theorems you use.
\end{itemize}
\end{problem}

\begin{answer}
%You can put your answer here.
\end{answer}





\begin{problem}  % problem 5
Let $S$ be a subset of $\R$.  Recall that $S$ is said to be {\it dense\/} in $\R$ if
for any open interval $(a,b)$, the intersection of $S$ with the set
$(a,b)$ is not empty.  (That is, there is at least one $s\in S$ such that $a < s < b$.)
Prove that $S$ is dense in $\R$ if and only if every point of $\R$ is an 
accumulation point of $S$.
\end{problem}

\begin{answer}
%You can put your answer here.
\end{answer}

 



\begin{problem}  % problem 6
Let $f(x)$ be a continuous function on a closed, bounded interval $[a,b]$.
In class, we used uniform continuity of $f$ to show that $f$ is bounded above.
However, it is possible to prove that directly using the Heine-Borel Theorem.
Follow this outline to prove that there is a number $M$ such that $f(x)\le M$ for
all $x\in[a,b]$:
\begin{itemize}
\item Show that for any $z\in[a,b]$, there is a $\delta_z>0$ and
a number $M_z$ such that $f(x)\le M_z$ for all $x\in(z-\delta_z,z+\delta_z)$.
(This is an easy consequence of continuity.  Just let $\eps=1$ in the definition
of continuity at $z$, and get $f(x)<f(z)+1$ for all $x$ near enough to $z$.)
\item Define an open cover of $[a,b]$ consisting of the
intervals $(z-\delta_z,z+\delta_z)$, for all $z\in[a,b]$.
(State why it is a cover.)
\item Apply the Heine-Borel Theroem, and finish the proof.
\end{itemize}
\end{problem}

\begin{answer}
%You can put your answer here.
\end{answer}






\begin{problem}  % problem 7
Suppose that $f(x)$ and $g(x)$ are uniformly continuous
on the interval $I$ (which is not necessarily closed or bounded).
Show directly from the definition of uniform continuity that $f(x)+g(x)$ is
uniformly continuous on $I$.
\end{problem}

\begin{answer}
%You can put your answer here.
\end{answer}











\end{document}



