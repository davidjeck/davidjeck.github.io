\documentclass[12pt]{article}
\usepackage[utf8]{inputenc}
\usepackage[letterpaper]{geometry}
\usepackage{amsmath}
\usepackage{amsfonts}
\usepackage{mathrsfs}
\usepackage{amsthm}
\usepackage{amssymb}
\usepackage{color}
\setlength{\topmargin}{-0.4 true in}
\setlength{\headsep}{0 true in}
\setlength{\topskip}{0 true in}
\setlength{\textwidth}{6.5 true in}
\setlength{\oddsidemargin}{0 true in}
\setlength{\evensidemargin}{0 true in}
\setlength{\textheight}{9.5 true in}
\renewcommand{\arraystretch}{1.2}
\newcommand{\eps}{\varepsilon}
\newcommand{\R}{{\mathbb R}}
\newcommand{\N}{{\mathbb N}}
\newcommand{\Q}{{\mathbb Q}}
\newcommand{\Z}{{\mathbb Z}}
\theoremstyle{definition}
\newtheorem{problem}{Problem}
\definecolor{darkblue}{RGB}{0,0,150}
\definecolor{darkred}{RGB}{180,0,0}
\definecolor{darkgreen}{RGB}{0,120,0}
\newcounter{partcounter}
\newcommand{\dspparts}[1]{\medskip
   \halign{\hskip40pt\stepcounter{partcounter}\llap{\bfseries\alph{partcounter})\ }$\displaystyle ##$\qquad\hfil&&
            \hskip30pt\stepcounter{partcounter}\llap{\bfseries\alph{partcounter})\ }$\displaystyle ##$\qquad\hfil\cr
   #1\crcr}}
\newenvironment{answer}{\par\medskip\bgroup\color{darkblue}}{\egroup\par\medskip}
\newenvironment{grading}{\par\medskip\bgroup\color{darkred}}{\egroup\par\bigskip}
\newenvironment{secondans}{\bgroup\color{darkgreen}}{\egroup}
\newcommand{\grd}[1]{{\color{darkred}#1}}
\pagestyle{empty}
\allowdisplaybreaks

\begin{document}

\hbox to \hsize{\textbf{Math 331, Fall 2022\hfill Homework 1}}
\nointerlineskip
\vskip 2pt
\hrule height 1pt

\bigskip

\centerline{\textit{This homework is due by the end of the day on Wednesday, August 31.}}
\centerline{\textit{It covers Sections 1.1 and 1.2 from the textbook.}}

\bigskip


\begin{problem}[Exercise 1.1.12]     % problem 1
Prove that if $a$ is irrational, then $\sqrt{a}$ is also irrational.
\end{problem}

\begin{answer}
% You should put your answer for proglem 1 here.  2 points.
\end{answer}



\begin{problem}[Exercises 1.1.14]   % problem 2
Show that $\sqrt{3}+\sqrt{2}$ is irrational as follows: First, show that if $\sqrt{3}+\sqrt{2}$
is rational then so is $\sqrt{3}-\sqrt{2}$.  (Hint: Consider their product.) Second, show that
$\sqrt{3}+\sqrt{2}$ and $\sqrt{3}-\sqrt{2}$ cannot both be rational. (Hint: Consider their sum.)
\end{problem}

\begin{answer}
% You should put your answer for problem 2 here.  3 points.
\end{answer}



\begin{problem}    % problem 3
Determine whether each set is bounded above and if so find its least upper bound.
Remember to briefly explain your answers.  For $D$ and $E$, you will need to quote some 
well-know facts about the relevant infinite series.
   \begin{enumerate}
   \item[] $A=\{1-\frac{1}{n}\,|\,n\in\N\}$
   \item[] $B=\{1+\frac{1}{n}\,|\,n\in\N\}$
   \item[] $C=[2,9)$
   \item[] $D=\{1,1+\frac{1}{2},1+\frac{1}{2}+\frac{1}{4},1+\frac{1}{2}+\frac{1}{4}+\frac{1}{8},1+\frac{1}{2}+\frac{1}{4}+\frac{1}{8}+\frac{1}{16},\dots\}$
   \item[] $E=\{1,1+\frac{1}{2},1+\frac{1}{2}+\frac{1}{3},1+\frac{1}{2}+\frac{1}{3}+\frac{1}{4},1+\frac{1}{2}+\frac{1}{3}+\frac{1}{4}+\frac{1}{5},\dots\}$
   \end{enumerate}
\end{problem}

\begin{answer}
% You should put your answer for problem 3 here.   5 points.
\end{answer}



\begin{problem}[From exercise 1.2.6]   % problem 4
Let $A$ and $B$ be arbitrary non-empty, bounded-above sets of real numbers.  Define
$C=\{a+b\,|\,a\in A$ and $b\in B\}$.  [That is, $C$ contains contains all sums
made up of one number from $A$ and one number from $B$.]
    \begin{enumerate}
    \item[(a)] Suppose that $\mu_1$ is an upper bound for $A$ and $\mu_2$ is an upper bound
        for $B$. Let $\mu = \mu_1+\mu_2$.  Show that is an upper bound for $C$.
    \item[(b)] Now suppose that $\lambda_1$ is the least upper bound for $A$ and
        $\lambda_2$ is the least upper bound for $B$.  Let $\lambda=\lambda_1+ \lambda_2$.
        Show that $\lambda$ is the least upper bound for $C$. (Hint: Use the
        last theorem in the third reading guide:  Let
        $\eps>0$.  Explain why there is an $a_o\in A$ such that $a_o> \lambda_1-\frac{\eps}{2}$
        and a $b_o\in B$ such that $b_o>\lambda_2-\frac{\eps}{2}$.  Use this to show
        $a_o+b_o>\lambda-\eps$, and conclude that $\lambda$ is the least upper bound for $C$.)
    \end{enumerate}
\end{problem}

\begin{answer}
% You should put your answer for problem 3 here.   4 points.
\end{answer}



\begin{problem}[From exercise 1.2.4]  %problem 5
Consider two sequences of real numbers $A=\{a_1,a_2,a_3,\dots\}$ and $B=\{b_1,b_2,b_3,\dots\}$,
which are bounded above.  Let $C$ be the set $C=\{a_1+b_1,a_2+b_2,a_3+b_3,\dots\}$.
[Compare this to the previous problem,
where $C$ contains only the sums of all elements of $A$ with all elements of $B$;
the $C$ in this problem contains only sums of corresponding elements from the two sequences.]
    \begin{enumerate}
    \item[(a)] Suppose that $\mu_1$ is an upper bound for $A$ and $\mu_2$ is an upper bound
        for $B$.  Show that $\mu_1+\mu_2$ is an upper bound for $C$.
    \item[(b)] Now suppose that $\lambda_1$ is the least upper bound for $A$ and
        $\lambda_2$ is the least upper bound for $B$.  Give an example to show that
        $\lambda_1+\lambda_2$ is not necessarily the least upper bound of $C$.
        [Hint: Take part (c) into account as you look for an example!]
    \item[(c)] Show that if $A$ and $B$ are \textit{non-decreasing} sequences, then
        $\lambda$ is in fact the least upper bound of $C$.  (Non-decreasing here means
        $a_1\le a_2\le a_3\le\cdots$ and $b_1\le b_2\le b_3\le\cdots$.)
    \end{enumerate}
\end{problem}

\begin{answer}
% You should put your answer for problem 5 here.   4 points.
\end{answer}



\begin{problem}  %problem 6
The last theorem in the third reading guide is about least upper bounds.  State the
corresponding theorem for greatest lower bounds.  You do not have to prove the theorem.
\end{problem}

\begin{answer}
% You should put your answer for problem 6 here.  2 points.
\end{answer}



\begin{problem}[Exercises 1.2.17 aamd 1.2.18]\  %problem 7
   \begin{enumerate}
   \item[(a)] Prove that the intersection of two Dedekind cuts is again a Dedekind cut.
   \item[(b)] Show that the intersection of an infinite number of Dedekind cuts is
     not necessarily a Dedekind cut, even if the intersection is non-empty, by using
     the following example:  For $n\in\N$, let $S_n$ be the Dedekind cut corresponding
     to the number $\frac{1}{n}$.  You need to show that
     $\bigcap_{n=1}^\infty\,S_n$ is not a Dedekind cut.
    \end{enumerate}
\end{problem}

\begin{answer}
% You should put your answer for problem 7 here.  4 points.
\end{answer}



\end{document}

