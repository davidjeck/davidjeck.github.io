\documentclass[12pt]{article}
\usepackage[utf8]{inputenc}
\usepackage[letterpaper]{geometry}
\usepackage{amsmath}
\usepackage{amsfonts}
\usepackage{mathrsfs}
\usepackage{amsthm}
\usepackage{amssymb}
\usepackage{color}
\setlength{\topmargin}{-0.2 true in}
\setlength{\headsep}{0 true in}
\setlength{\topskip}{0 true in}
\setlength{\textwidth}{6.5 true in}
\setlength{\oddsidemargin}{0 true in}
\setlength{\evensidemargin}{0 true in}
\setlength{\textheight}{8.8 true in}
\renewcommand{\arraystretch}{1.2}
\newcommand{\eps}{\varepsilon}
\newcommand{\R}{{\mathbb R}}
\newcommand{\N}{{\mathbb N}}
\newcommand{\Q}{{\mathbb Q}}
\newcommand{\Z}{{\mathbb Z}}
\theoremstyle{definition}
\newtheorem{problem}{Problem}
\definecolor{darkblue}{RGB}{0,0,150}
\definecolor{darkred}{RGB}{180,0,0}
\newenvironment{answer}{\par\medskip\bgroup\color{darkblue}}{\egroup\par\medskip}
\newenvironment{grading}{\par\medskip\bgroup\color{darkred}}{\egroup\par\bigskip}
\pagestyle{empty}
\allowdisplaybreaks


\begin{document}

\hbox to \hsize{\textbf{Math 331, Fall 2022\hfill Take-home Test \#2}}
\nointerlineskip
\vskip 3pt
\hrule height 1pt

\medskip

{\narrower\narrower\small{
This test is due by the start of class on the day of the second in-class test, Friday, November 18.  
\textbf{You should do any six out of the seven problems on the test.  Do not do
all seven; if you do all seven, your answer for problem~7 will be ignored.}\par
You should do this test on your own, using only your textbook, class notes, 
and previous work in the course as reference.  You should
not work with other students, you should not use the Internet or other references, and you should not
consult anyone except the professor for the course.  You can ask questions about the test in class and by email.
I might give some hints and clarifications, but I am unlikely to give
extensive, detailed help.   Be sure to show all of
your work!  If you are not able to complete a problem, you should turn in work showing
whatever progress you have made on it, for partial credit.
\textbf{Note: There will be no ``rewrites'' for this test.}\par
You can submit your answers in LaTeX on overleaf.com, if you want to do 
that, but you are welcome to write up your answers neatly by hand.\par
Although some problems are more difficult than others, and some are fairly easy, 
all seven problems will count equally.}
\par}

\medskip
\hrule height 1pt
\bigskip


\def\ds{\displaystyle}


\begin{problem} % problem 1
\textbf{(a)} Suppose that the function $F(x)$ is differentiable at $a$.  Show directly from the definition
of derivative that the function $G(x) = F(x)^2$ is differentiable at $a$ and $G'(a)=2F(a)F'(a)$.
[Hint: You only need to factor $F(x)^2-F(a)^2$ in the definition.]

\textbf{(b)} We know that $f(x)g(x) = \frac{1}{4}\big((f(x)+g(x))^2-(f(x)-g(x))^2\big)$ from a previous homework problem .
Using only this fact, the result from part (a), and the sum, difference, and constant multiple rules for derivatives,
find the formula for the derivative of $f(x)g(x)$, 
\end{problem}

\begin{answer}
%You can put your answer here.
\end{answer}





\begin{problem} % problem 2
Let $f$ and $g$ be differentiable functions on $[a,b]$.  Suppose that $f(a)=g(a)$ and $f'(x)>g'(x)$ for all
$x\in(a,b)$.  Prove that $f(b)>g(b)$.  [Hint:  Consider the function $h(x)=f(x)-g(x)$ and apply the Mean
Value Theorem.]
\end{problem}

\begin{answer}
%You can put your answer here.
\end{answer}





\begin{problem} % problem 3
Let $f$ be an integrable function on $[a,b]$.  Suppose that $A\le f(x) \le B$ for all $x\in [a,b]$.
Show, from the definition of the integral, that $A\cdot (b-a)\le\int_a^b f\le B\cdot (b-a)$.
(Hint: Use the trivial partition $P=\{x_0,x_1\}$ where $x_0=a$, $x_1=b$.)
\end{problem}

\begin{answer}
%You can put your answer here.
\end{answer}





\begin{problem} % problem 4
Suppose that $f$ is integrable on $[a,b]$.  
Define $F(x)=\int_a^x f$ for $x\in[a,b]$,
and define $G(x)=\int_a^x F$ for $x\in[a,b]$.  How do we know $\int_a^x F$ exists?  Show that
$G$ is differentiable on $[a,b]$.
\end{problem}

\begin{answer}
%You can put your answer here.
\end{answer}





\begin{problem} % problem 5
Let $\sum_{k=1}^\infty a_k$ be a convergent series of non-negative terms.
Prove that the series $\sum_{k=1}^\infty a_k^2$ also converges.
[Hints: $\big(\frac{1}{2}\big)^2=\frac{1}{4}$, and remember that you only
need to show $\sum_{k=N}^\infty a_k^2$ converges for some $N$.]
\end{problem}

\begin{answer}
%You can put your answer here.
\end{answer}




\begin{problem} % problem 6
[\it Textbook problem 4.5.7, 8]
\textbf{(a)}\ Let $\{f_n\}_{n=1}^\infty$
be a sequence of functions defined on an interval $I$. Assume that each $f_n$ is bounded; that is,
there are constants $M_n$ such that $|f_n(x)|\le M_n$ for all $x\in I$. Prove: If 
$\{f_n\}_{n=1}^\infty$ converges uniformly to $f$, then $f$ must also be bounded on $I$.

\textbf{(b)}\ Show that the hypothesis of uniform convergence is necessary
by finding a sequence of bounded functions that converges pointwise to a function that
is not bounded.  ([Hint: Take $I=[0,\infty)$ and look for a simple example.]
\end{problem}

\begin{answer}
\end{answer}



\begin{problem} % problem 7
Suppose that the function $f:\R\to\R$ satisfies $|f(x)-f(y)|\le r|x-y|$ for all $x,y\in\R$,
where $r$ is a constant in the interval $0\le r<1$.  Such a function is said to be a 
\textbf{contraction} on $\R$.  Note that a contraction is simply a Lipschitz function with
Lipschitz constant strictly less than 1, so we already know that $f$ is continuous.
\begin{itemize}
\item[\bf(a)]
   Let $t$ be any real number.  Define a sequence $\{a_n\}_{n=0}^\infty$ by 
   $a_0 = t$, $a_n = f(a_{n-1})$ for $n>0$.   That is $a_0 = t, a_1 = f(t), a_2=f(f(t)), 
   a_3=f(f(f(t))), \dots, a_n = f^n(t),\dots$, where $f^n$ is the composition of
   $f$ with itself $n$ times.  Show that the sequence $\{a_n\}_{n=0}^\infty$ is contracting,
   and hence is convergent.
\item[\bf(b)]
   Let $z = \ds\lim_{n\to\infty} a_n$.  Show that $f(z) = z$, that is, $z$ is a fixed point of $f$.
   [Hint: Write $f(z)=f\big(\ds\lim_{n\to\infty}a_n\big)=f\big(\lim_{n\to\infty}f^n(t)\big)$. and
   use the fact that $f$ is continuous.]
\end{itemize}
(Note: Recall that a \textbf{fixed point} of a function $f$ is a point $y$ such that $f(y) = y$.
It is clear that a contraction can have at most one fixed point.  This problem shows that
a contraction always does have a fixed point.  Furthermore, if $t$ is any real number, then
the sequence $\{f^n(t)\}_{n=0}^\infty$ converges to that unique fixed point.  This is the
\textbf{Contraction Mapping Theorem} for $\R$.)
\end{problem}

\begin{answer}
%You can put your answer here.
\end{answer}





%\begin{problem} % problem 7
%Let $\{x_n\}_{n=1}^\infty$ be an infinite sequence.  We define a
%{\it rearrangement\/} of the sequence as follows: Let $s\colon \N\to\N$ be a
%bijective function.  Then $\{x_{s(i)}\}_{i=1}^\infty$ is a rearrangement of the
%sequence $\{x_n\}_{n=1}^\infty$.  The rearranged sequence has exactly the same terms 
%as the original sequence, just in a different order.
%
%Suppose that $\{x_n\}_{n=1}^\infty$ is a convergent sequence and that
%$\ds\lim_{n\to\infty}x_n=L$, and let $\{x_{s(i)}\}_{i=1}^\infty$ be a rearrangement of
%the sequence.  Show that the rearranged sequence $\{x_{s(i)}\}_{i=1}^\infty$ is
%convergent and converges to the same limit, $\ds\lim_{i\to\infty}x_{s(i)}=L$.
%[Hint: This is easier than it looks.  Note that for any $N\in\N$, the set $\{x_1,x_2,\dots,x_N\}$ is
%finite.  Use the definition of convergence.]
%\end{problem}
%
%\begin{answer}
%You can put your answer here.
%\end{answer}
%








\end{document}



