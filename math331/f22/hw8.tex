\documentclass[12pt]{article}
\usepackage[utf8]{inputenc}
\usepackage[letterpaper]{geometry}
\usepackage{amsmath}
\usepackage{amsfonts}
\usepackage{mathrsfs}
\usepackage{amsthm}
\usepackage{amssymb}
\usepackage{color}
\setlength{\topmargin}{-0.4 true in}
\setlength{\headsep}{0 true in}
\setlength{\topskip}{0 true in}
\setlength{\textwidth}{6.5 true in}
\setlength{\oddsidemargin}{0 true in}
\setlength{\evensidemargin}{0 true in}
\setlength{\textheight}{9.7 true in}
\renewcommand{\arraystretch}{1.2}
\newcommand{\eps}{\varepsilon}
\newcommand{\R}{{\mathbb R}}
\newcommand{\N}{{\mathbb N}}
\newcommand{\Q}{{\mathbb Q}}
\newcommand{\Z}{{\mathbb Z}}
\theoremstyle{definition}
\newtheorem{problem}{Problem}
\allowdisplaybreaks
\definecolor{darkblue}{RGB}{0,0,150}
\definecolor{darkred}{RGB}{180,0,0}
\definecolor{darkgreen}{RGB}{0,120,0}
%\newenvironment{answer}{\par\medskip\bgroup\color{darkblue}\noindent\textbf{Answer:}\par\medskip}{\egroup\medskip}
\newenvironment{answer}{\par\medskip\bgroup\color{darkblue}}{\egroup\medskip}
\newenvironment{grading}{\par\medskip\bgroup\color{darkred}}{\egroup\medskip}
\newenvironment{secondans}{\par\medskip\bgroup\color{darkgreen}}{\egroup\medskip}
\newcommand{\grd}[1]{{\color{darkred}#1}}
\pagestyle{empty}


\begin{document}

\hbox to \hsize{\textbf{Math 331, Fall 2022\hfill Homework 8}}
\nointerlineskip
\vskip 2pt
\hrule height 1pt

\medskip

\centerline{\textit{This homework is due on Wednesday, November 2.}}
\smallskip
\centerline{\textit{Note that problems 1 and 2 are not proofs and are not eligible for revision.}}

\bigskip

\def\ds{\displaystyle}

\def\seq#1{\{#1_n\}_{n=1}^\infty}
\def\slim#1{{\ds\lim_{n\to\infty} #1_n}}

\begin{problem}  % problem 1, 3 points
Let $f$ be the polynomial $f(x)=2-5x^2+3x^3-x^4$.  Use Taylor's Theorem to write $f$ as a
polynomial in powers of $(x+1)$.  (That is, find the Taylor polynomial, $p_{4,-1}(x)$, of degree 4 at $-1$ for $f$.)
\end{problem}

\begin{answer}
%You can put your answer here for problem 1
\end{answer}




\begin{problem}  % problem 2, 3 points
Find the general Taylor polynomial at 0, $p_{n,0}(x)$, for the function $\ln(x+1)$.
\end{problem}

\begin{answer}
%You can put your answer here for problem 2
\end{answer}




\begin{problem}[from 4.2.14 from the textbook]    % problem 3, 4 points
We have shown that the $n^{th}$ Taylor polynomial for $e^x$ at 0 is
$p_{n,0}(x)=\sum_{k=1}^n\frac1{n!}x^n$.
Show that $e$ is irrational by using proof by contradiction.  Suppose,
for the sake of contradiction, that $e=\frac pq$ for some integers $p$ and $q$.

\smallskip
\textbf{(a)}\ \ Use the Lagrange form of the remainder term from Taylor's Theorem to show that there is a $c\in[0,1]$
such that $\frac pq-\big(\frac1{0!}+\frac1{1!}+\cdots+\frac1{n!}\big)=\frac{e^c}{(n+1)!}$.

\smallskip
\textbf{(b)}\ \ Multiply both sides of the equation in (a) by $n!$, and show that left side of the
resulting equation is an integer when $n\ge q$.

\smallskip
\textbf{(c)}\ \  Show that the right side of the equation that
you got in part (b) is not an integer when $n>e$.  Conclude that $e$ is irrational.
\end{problem}

\begin{answer}
%You can put your answer here for problem 3
\end{answer}




\begin{problem} % problem 4, 2 points
Suppose that $f$ is a function defined for all $x\ge 1$ and that $\ds\lim_{x\to+\infty}f(x)=L$.
Define a sequence $\seq{a}$ by $a_n=f(n)$ for all $n\in\N$.  Prove that $\slim{a}=L$.
\end{problem}

\begin{answer}
%You can put your answer here for problem 3
\end{answer}




\begin{problem}  % problem 5, 3 points
Prove that if $\seq{x}$ is an increasing sequence that is not bounded above,
then $\slim{x} = +\infty$.
\end{problem}

\begin{answer}
%You can put your answer here for problem 5
\end{answer}




\begin{problem}[From Textbook problem 4.2.5]  % problem 6, 3 points
Let $\seq{a}$ be defined inductively as follows: 
$$a_1=1,\qquad\qquad a_n = 1 + \frac{a_{n-1}}{4}\mbox{\ \ for }n>1$$
\begin{itemize}
\item[\bf(a)]
   Show by induction that $a_n$ is bounded above by $4/3$.
\item[\bf(b)]
    Show that $\seq a$ is convergent by showing that it is increasing.
\item[\bf(c)]
    Show that $\slim{a} = 4/3$. [Hint: Use the fact that
    $\slim{a} = \ds\lim_{n\to\infty}a_{n+1}$ and the recursive
    definition of $a_n$.]
\end{itemize}
\end{problem}

\begin{answer}
%You can put your answer here for problem 6
\end{answer}





\begin{problem}  % problem 7, 3 points
Suppose that $\seq{a}$ and $\seq{b}$ are sequences, and $\seq{a}$ is convergent with $\slim{a} = L$.
Suppose in addition that $\ds\lim_{n\to\infty}|a_n-b_n| = 0$.  Show that $\seq{b}$ is convergent
and $\slim{b} = L$.
\end{problem}

\begin{answer}
%You can put your answer here for problem 7
\end{answer}





\end{document}
