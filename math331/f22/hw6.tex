\documentclass[12pt]{article}
\usepackage[utf8]{inputenc}
\usepackage[letterpaper]{geometry}
\usepackage{amsmath}
\usepackage{amsfonts}
\usepackage{mathrsfs}
\usepackage{amsthm}
\usepackage{amssymb}
\usepackage{color}
\setlength{\topmargin}{-0.5 true in}
\setlength{\headsep}{0 true in}
\setlength{\topskip}{0 true in}
\setlength{\textwidth}{6.5 true in}
\setlength{\oddsidemargin}{0 true in}
\setlength{\evensidemargin}{0 true in}
\setlength{\textheight}{9.5 true in}
\renewcommand{\arraystretch}{1.2}
\newcommand{\eps}{\varepsilon}
\newcommand{\R}{{\mathbb R}}
\newcommand{\N}{{\mathbb N}}
\newcommand{\Q}{{\mathbb Q}}
\newcommand{\Z}{{\mathbb Z}}
\theoremstyle{definition}
\newtheorem{problem}{Problem}
\definecolor{darkblue}{RGB}{0,0,150}
\definecolor{darkred}{RGB}{180,0,0}
\definecolor{darkgreen}{RGB}{0,120,0}
%\newenvironment{answer}{\par\medskip\bgroup\color{darkblue}\noindent\textbf{Answer:}\par\medskip}{\egroup\medskip}
\newenvironment{answer}{\par\medskip\bgroup\color{darkblue}}{\egroup\medskip}
\newenvironment{grading}{\par\medskip\bgroup\color{darkred}}{\egroup\medskip}
\newenvironment{secondans}{\par\medskip\bgroup\color{darkgreen}}{\egroup\medskip}
\newcommand{\grd}[1]{{\color{darkred}#1}}
\pagestyle{empty}


\begin{document}

\hbox to \hsize{\textbf{Math 331, Fall 2022\hfill Homework 6}}
\nointerlineskip
\vskip 2pt
\hrule height 1pt

\medskip

\centerline{\textit{This homework is due on Monday, October 17.}}

\bigskip

\def\ds{\displaystyle}

\begin{problem} % problem 1
Let $(M,d)$ be a metric space, and let $f\colon M\to \R$ and
$g\colon M\to \R$ be two functions from $M$ to $\R$ (where $\R$ has
its usual metric).  Let $a\in M$.  Suppose $f$ and $g$ are
continuous at $a$.  Show that the function $f+g$ is continuous
at $a,$ where $(f+g)(x) = f(x) + g(x)$ for $x\in M.$
[Hint: Just imitate the proof for functions from $\R$ to $\R.$]
\end{problem}

\begin{answer}
%You can put your answer here, 3 points
\end{answer}



\begin{problem}  % problem 2
Let $X$ be any set. Consider the metric space $(X,\delta)$ where $\delta$ is the
discrete metric, $\delta\colon X\times X\to \R$ by $\delta(a,b) = 
\begin{cases} 0&\text{if }a=b\\ 1&\text{if }a\not=b\end{cases}$.
Suppose that $\{x_i\}_{i=1}^\infty$ is a \textbf{convergent} sequence in the
metric space $(X,\delta).$  Show that there is a number $N$ such
that $x_N=x_{N+1}=x_{N+2}=\cdots$.  (We say that the sequence is
``eventually constant.'') [Hint: The number is the limit of the sequence.]

\end{problem}

\begin{answer}
%You can put your answer here, 3 points
\end{answer}



\begin{problem}  % problem 3
Let $(M,d)$ be a metric space and let $X\subseteq M$. The closure, $\overline{X}$ of $X$ can be
defined as the set containing all the points of $X$ plus all the accumulation
points of $X$.  Show that the
closure of $X$ can be characterized as follows:  For $z\in M,$ $z \in \overline{X}$ if and only if there is a
sequence $\{x_n\}_{n=1}^\infty$ of points of $X$ such that $\displaystyle \lim_{n\to\infty}x_n = z.$
[Hint: Treat separately the cases where $z\in X$ and where $z$ is an accumulation
point of $X.$]
\end{problem}


\begin{answer}
%You can put your answer here, 3 points
\end{answer}



\begin{problem}  % problem 4
[From textbook problem 3.1.3a]
Even though $|x|$ is not differentiable at 0, show that the function $g(x)=\frac{1}{2}x|x|$ is
differentiable at~0, and show that $g'(x)=|x|$ for all $x$.  (Thus, $|x|$ has antiderivative
$\frac{1}{2}x|x|$.)
\end{problem}

\begin{answer}
%You can put your answer here, 3 points
\end{answer}



\begin{problem}  % problem 5
[Textbook problem 3.3.10]
A \textbf{fixed point} of a function is a point $d$ such that $f(d)=d$.  Suppose that
$f$ is differentiable everywhere and that $f'(x)<1$ for all $x$.  Show that there can
be at most one fixed point for $f$. [Hint:  Suppose that $a$ and $b$ are two fixed points of $f$.
Apply the Mean Value Theorem to obtain a contradiction.]
\end{problem}

\begin{answer}
%You can put your answer here, 3 points
\end{answer}



\begin{problem} % problem 6
[From textbook problem 3.3.2]
Recall that $f$ satisfies a Lipschitz condition if there is a constant $M$ such that
$|f(b)-f(a)|\le M|b-a|$ for all $a,b$.  Problem \#9 on Homework \#4 proved that
any function that satisfies a Libschitz condition is uniformly continuous.
Let $f$ be a function that is differentiable on some interval I (not necessarily bouned or closed),
and suppose $f'(x)\le M$ for all $x$, where $M$ is some constant.  Use the 
Mean Value Theorem to prove that $|f(b)-f(a)|\le M|a-b|$ for all $a,b$.
Conclude that $f$ is uniformly continuous.
\end{problem}

\begin{answer}
%You can put your answer here, 3 points
\end{answer}





\end{document}
