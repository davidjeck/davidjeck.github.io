\documentclass[11pt]{article}
\usepackage[utf8]{inputenc}
\usepackage[letterpaper]{geometry}
\usepackage{amsmath}
\usepackage{amsfonts}
\usepackage{mathrsfs}
\usepackage{amsthm}
\usepackage{amssymb}
\usepackage{color}
\setlength{\topmargin}{-0.5 true in}
\setlength{\headsep}{0 true in}
\setlength{\topskip}{0 true in}
\setlength{\textwidth}{6.5 true in}
\setlength{\oddsidemargin}{0 true in}
\setlength{\evensidemargin}{0 true in}
\setlength{\textheight}{9.7 true in}
\renewcommand{\arraystretch}{1.2}
\newcommand{\eps}{\varepsilon}
\newcommand{\R}{{\mathbb R}}
\newcommand{\N}{{\mathbb N}}
\newcommand{\Q}{{\mathbb Q}}
\newcommand{\Z}{{\mathbb Z}}
\theoremstyle{definition}
\newtheorem{problem}{Problem}
\definecolor{darkblue}{RGB}{0,0,150}
\definecolor{darkred}{RGB}{180,0,0}
\definecolor{darkgreen}{RGB}{0,120,0}
\newcounter{partcounter}
\newcommand{\dspparts}[1]{\medskip
   \halign{\hskip40pt\stepcounter{partcounter}\llap{\bfseries\alph{partcounter})\ }$\displaystyle ##$\qquad\hfil&&
            \hskip30pt\stepcounter{partcounter}\llap{\bfseries\alph{partcounter})\ }$\displaystyle ##$\qquad\hfil\cr
   #1\crcr}}
%\newenvironment{answer}{\par\medskip\bgroup\color{darkblue}\noindent\textbf{Answer:}\par\medskip}{\egroup\medskip}
\newenvironment{answer}{\par\medskip\bgroup\color{darkblue}}{\egroup\medskip}
\newenvironment{grading}{\par\medskip\bgroup\color{darkred}}{\egroup\medskip}
\newenvironment{secondans}{\par\medskip\bgroup\color{darkgreen}}{\egroup\medskip}
\newcommand{\grd}[1]{{\color{darkred}#1}}
\pagestyle{empty}
\allowdisplaybreaks


\begin{document}

\hbox to \hsize{\textbf{Math 331, Fall 2022\hfill Homework 9}}
\nointerlineskip
\vskip 2pt
\hrule height 1pt

\medskip

{\narrower\narrower \textit{
   This homework is due on Friday, November 11.  At that time, the second take-home 
   test will be avaialble, and that test will be due at the start of class on 
   Friday, November 18.  The second in-class test will be given on November 18.
   Resubmissions for Homework 9 will be due on November 21.  There might be
   one more homework assignment, which will be due on the last day of class.
   Note that this homework is worth a total of 32 points.
}\par}

\bigskip

\def\ds{\displaystyle}

\def\seq#1{\{#1_n\}_{n=1}^\infty}
\def\slim#1{{\ds\lim_{n\to\infty} #1_n}}

\begin{problem}  % problem 1, 2 points
A previous homework problem showed that $\frac{1}{2}x|x|$ is an antiderivative for $|x|$.  Using that
fact, evaluate $\int_{-3}^5 |x|\,dx$ \textbf{using the first Fundamental Theorem of Calculus once}.  Explain
why the answer makes sense geometrically (in terms of area).
\end{problem}

\begin{answer}
%You can put your answer here.
\end{answer}


\begin{problem}  % problem 2, 3 points
\ \ {\bf (a)}\ \ Suppose that $\sum_{k=1}^\infty a_k$ is a convergent series, and $\sum_{k=1}^\infty b_k$ 
is a divergent series.  Show that $\sum_{k=1}^\infty(a_k+b_k)$ diverges.  [Hint:  Proof by contradiction 
will work.]

\smallskip
\ \ {\bf (b)}\ \ Suppose that $\sum_{k=1}^\infty a_k$ and $\sum_{k=1}^\infty b_k$
are both divergent.  Give a simple example to show that $\sum_{k=1}^\infty(a_k+b_k)$ does not 
necessarily diverge.
\end{problem}

\begin{answer}
%You can put your answer here.
\end{answer}




\begin{problem}  % problem 3, 16 points
A nonnegative series must either converge (absolutely) or diverge to $+\infty$.  Classify each of the
following nonnegative series as either convergent or divergent.  In all cases, explain your reasoning, being 
explicit about any convergence tests that you apply.

\bigskip
\dspparts{
   \sum_{k=1}^\infty \frac{3k^2}{7k^5 + 2k^2} &
   \sum_{k=1}^\infty \frac{k}{\sqrt{k^2+1}} &
   \sum_{n=0}^\infty \pi^{-n} &
   \sum_{n=1}^\infty \frac{2^n+3^n}{4^n} \cr\noalign{\bigskip\medskip}
   \sum_{m=1}^\infty \frac{1+\sin(m)}{5^m} &
   \sum_{n=1}^\infty \frac{n^6}{5^n} &
   \sum_{k=1}^\infty \frac{(k!)^2}{(2k)!} &
   \sum_{n=1}^\infty \frac{1}{n+\sqrt{n}}
}
\end{problem}

\begin{answer}
%You can put your answer here.
\end{answer}




\begin{problem}  % problem 4, 8 points
A series of positive and negative terms can either diverge, converge absolutely, or converge conditionally.
Classify each of the following series as one of divergent, absolutely convergent, or conditionally convergent.
In all cases, explain your reasoning, being explicit about any convergence tests that you apply.

\setcounter{partcounter}{0}
\bigskip
\dspparts{
    \sum_{n=0}^\infty\frac{(-1)^{n+1}}{3^n} &
    \sum_{k=1}^\infty\frac{(-1)^k}{\sqrt{k+1}} \cr\noalign{\bigskip\medskip}
    \sum_{n=2}^\infty\frac{(-1)^n\ln(n)}{n} &
    \sum_{k=1}^\infty\frac{(-1)^k}{k^k}   
}
\end{problem}

\begin{answer}
%You can put your answer here/
\end{answer}




\begin{problem}  % problem 5, 3
The series $\sum_{k=2}^\infty \frac{1}{k\ln(k)}$ diverges, but none of the tests\vadjust{\smallskip}
that we have covered can prove it.  Note that ${\ds\lim_{n\to\infty}}\big(\int_2^n\frac{1}{x\ln(x)}dx\big)
={\ds\lim_{n\to\infty}}\big(\ln(\ln(x))-\ln(\ln(2))\big) = +\infty$.  Also\vadjust{\smallskip} note that
$f(x) = \frac{1}{x\ln(x)}$ is decreasing.  [You do not have to prove these facts.]
Show that\vadjust{\smallskip} the partial sum,
$s_n=\sum_{k=2}^n\frac{1}{k\ln(k)}$, satisfies $$s_n\ge \int_2^{n+1}\frac{1}{x\ln(x)}\,dx$$
by considering the upper sum using the partition $\{2,3,4,\dots,n+1\}$ of the interval $[2,n+1]$,
and conclude that $\sum_{k=2}^\infty \frac{1}{k\ln(k)}$ diverges.  (Note that this example is a special
case of something called the ``integral test.'')
\end{problem}

\begin{answer}
%You can put your answer here.
\end{answer}







\end{document}
