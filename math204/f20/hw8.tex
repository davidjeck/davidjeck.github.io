\documentclass[11pt]{article}
\usepackage[utf8]{inputenc}
\usepackage[letterpaper]{geometry}
\usepackage{amsmath}
\usepackage{amsfonts}
\usepackage{amsthm}
\usepackage{amssymb}
\usepackage{mathrsfs}
\usepackage{color}
\usepackage{graphicx}
\def\scaledgraphics#1#2{\includegraphics[width=#1]{#2}}
\setlength{\topmargin}{-0.4 true in}
\setlength{\headsep}{0 true in}
\setlength{\topskip}{0 true in}
\setlength{\textwidth}{6.5 true in}
\setlength{\oddsidemargin}{0 true in}
\setlength{\evensidemargin}{0 true in}
\setlength{\textheight}{9.5 true in}
\renewcommand{\arraystretch}{1.2}
\newcommand{\eps}{\varepsilon}
\newcommand{\R}{{\mathbb R}}
\newcommand{\N}{{\mathbb N}}
\newcommand{\Q}{{\mathbb Q}}
\newcommand{\Z}{{\mathbb Z}}
\theoremstyle{definition}
\newtheorem{problem}{Problem}
\definecolor{darkblue}{RGB}{0,0,150}
\definecolor{darkred}{RGB}{180,0,0}
\definecolor{darkgreen}{RGB}{0,120,0}
\newenvironment{answer}{\par\bigskip\bgroup\color{darkblue}}{\egroup}
\newenvironment{grading}{\par\medskip\bgroup\color{darkred}}{\egroup\par\bigskip}
\newenvironment{secondans}{\bgroup\color{darkgreen}}{\egroup}
\pagestyle{empty}


\def\rowop#1{\qquad\xrightarrow{\begin{matrix}#1\end{matrix}}\qquad}

\begin{document}


\hbox to \hsize{\bf Math 204, Fall 2020\hfil  Homework \#8}
\nointerlineskip
\vskip 2pt
\hrule height 1pt

\medskip

\centerline{\textit{This homework is due by 11:59 PM on Thursday, October 29}}

\medskip



\begin{problem}
Let $A$ be the matrix 
   $A=\begin{pmatrix}  
      1& 3& 0\\
      0& 0& 1\\
      2& 1& 0
   \end{pmatrix}$.  
Put the matrix $A$ into reduced echelon form.  This can be done with four row operations.
Now, based on your row reduction, write the matrix $A$ as a product of $3\times 3$ matrices, 
where each matrix in the product is an elementary matrix.
\end{problem}

\begin{answer}
% Answer here for problem number 1,  points 4
\end{answer}




\begin{problem}
The $n\times n$ identity matrix, $I_n$, has the property that it is its
own inverse.  That is, the product $I_nI_n$ is equal to $I_n$.  There are
other $n\times n$ matrices that have the same property; that is, $AA=I_n$.
\begin{itemize}
\item[\bf(a)] Describe all \textbf{diagonal} $n\times n$ matrices $D$ that 
   have the property $DD=I_n$.
\item[\bf(b)]  Let $S$ be the $2\times 2$ matrix $S=\begin{pmatrix}0&1\\1&0\end{pmatrix}$.  Calculate the matrix
   product $SS$ to see that $S$ is its own inverse. 
\item[\bf(c)]  The matrix $S$ from the previous part is a \textbf{permutation matrix};
multiplying a $2\times n$ matrix on the left by $S$ will swap the two rows of that matrix, so
$SS$ is the matrix that you get by swapping the rows of $S$, producing the identity 
matrix.  Find two different $3\times 3$ permutation matrices $A$ and $B$ that are their own inverses.
That is, $AA=I_3$ and $BB=I_3$.
\item[\bf(d)]  Find a $3\times 3$ permutation matrix $A$ that has the property $AAA=I_3$.
\end{itemize}
\end{problem}

\begin{answer}
% Answer here for problem number 2, 5 points
\end{answer}




\begin{problem}
Let $d\colon {\mathscr P}_4 \to {\mathscr P}_3$ be the derivative,
$d(p(x)) = p'(x)$.  Find the matrix ${\rm Rep}_{B,D}(d)$ where $B$ and $D$
are the usual bases for ${\mathscr P}_4$ and ${\mathscr P}_3$,
$B=\langle 1,x,x^2,x^3 \rangle$ and $D=\langle 1,x,x^2 \rangle$.
\end{problem}

\begin{answer}
% Answer here for problem number 3, 3 points
\end{answer}




\begin{problem}
Let $h$ be the homomorphism $h\colon {\mathscr P}_2 \to {\mathscr P}_2$ given by
$$h(a+bx+cx^2) = (a+b) + (b+c)x + (c+a)x^2$$
Let $B$ be the basis of $\mathscr P_2$ given by $B=\langle 1, 1+x, 1+x+x^2\rangle$.
Find the matrix ${\rm Rep}_{B,B}(h)$.
\end{problem}

\begin{answer}
% Answer here for problem number 4, 3 points
\end{answer}




\begin{problem}
Let $f\colon \R^3\to\R^2$ be the homomorphism given by $f\begin{pmatrix}a\\b\\c\end{pmatrix}=
\begin{pmatrix} 3a+b\\2b-c\end{pmatrix}$.  Find the matrix ${\rm Rep}_{B,D}$ where the bases
$B$ and $D$ of $\R^2$ and $\R^2$ are given by
$$B=\left\langle
    \begin{pmatrix} 1\\1\\1 \end{pmatrix},
    \begin{pmatrix} 0\\1\\1 \end{pmatrix},
    \begin{pmatrix} 0\\0\\1 \end{pmatrix}
  \right\rangle \mbox{\qquad and \qquad}
 D=\left\langle
    \begin{pmatrix} 1\\3 \end{pmatrix}, 
    \begin{pmatrix} 0\\-1 \end{pmatrix}
 \right\rangle$$
\end{problem}

\begin{answer}
% Answer here for problem number 5, 3 points
\end{answer}




\begin{problem}
Let $V$ be a vector space with basis $B=\langle \vec\beta_1,\vec\beta_2,\dots,\vec\beta_n \rangle$.
Let $g\colon V\to V$ is a homomorphism.  Suppose that there are numbers 
$\lambda_1,\lambda_2,\dots,\lambda_n\in\R$ such that $g(\vec\beta_1)=\lambda_1\cdot\vec\beta_1$,
$g(\vec\beta_2)=\lambda_2\cdot\vec\beta_2$, \dots, $g(\vec\beta_n)=\lambda_n\cdot\vec\beta_n$.
What is ${\rm Rep}_{B,B}(g)\,$?

(Preview:  If $h\colon V\to V$ is a homomorphism and $h(\vec v) = \lambda\cdot\vec v$ for
some $\lambda\in\R$ and $\vec v\in V$, then $\lambda$ is called an \textbf{eigenvalue} for $h$,
and $\vec v$ is called an \textbf{eigenvector} for $h$ with eigenvalue $\lambda$.  The homomorphism
$g$ in this problem admits a \textbf{basis of eigenvectors}, but this is not the usual case.)
\end{problem}

\begin{answer}
% Answer here for problem number 6, 2 points
\end{answer}




\end{document}

