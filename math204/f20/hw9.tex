\documentclass[11pt]{article}
\usepackage[utf8]{inputenc}
\usepackage[letterpaper]{geometry}
\usepackage{amsmath}
\usepackage{amsfonts}
\usepackage{amsthm}
\usepackage{amssymb}
\usepackage{mathrsfs}
\usepackage{color}
\usepackage{graphicx}
\def\scaledgraphics#1#2{\includegraphics[width=#1]{#2}}
\setlength{\topmargin}{-0.4 true in}
\setlength{\headsep}{0 true in}
\setlength{\topskip}{0 true in}
\setlength{\textwidth}{6.5 true in}
\setlength{\oddsidemargin}{0 true in}
\setlength{\evensidemargin}{0 true in}
\setlength{\textheight}{9.5 true in}
\renewcommand{\arraystretch}{1.2}
\newcommand{\eps}{\varepsilon}
\newcommand{\R}{{\mathbb R}}
\newcommand{\N}{{\mathbb N}}
\newcommand{\Q}{{\mathbb Q}}
\newcommand{\Z}{{\mathbb Z}}
\theoremstyle{definition}
\newtheorem{problem}{Problem}
\definecolor{darkblue}{RGB}{0,0,150}
\definecolor{darkred}{RGB}{180,0,0}
\definecolor{darkgreen}{RGB}{0,120,0}
\newenvironment{answer}{\par\bigskip\bgroup\color{darkblue}}{\egroup}
\newenvironment{grading}{\par\medskip\bgroup\color{darkred}}{\egroup\par\bigskip}
\newenvironment{secondans}{\bgroup\color{darkgreen}}{\egroup}
\pagestyle{empty}


\def\rowop#1{\qquad\xrightarrow{\begin{matrix}#1\end{matrix}}\qquad}

\begin{document}


\hbox to \hsize{\bf Math 204, Fall 2020\hfil  Homework \#9}
\nointerlineskip
\vskip 2pt
\hrule height 1pt

\medskip

\centerline{\textit{This homework is due by 11:59 PM on Tuesday, November 10}}

\medskip



\begin{problem}
Remember that the change of basis matrix for bases $B$ and $D$ of the same
vector space $V$ is defined to be $Rep_{B,D}(id)$, where $id\colon V\to V$
is the identity map.  

Suppose that $B$ and $D$ are bases for $\mathscr P_2$, where $D=
   \langle x+x^2, 1-2x+x^2, 2-x \rangle$.
Find the basis $B$ if the change of basis matrix is
${\rm Rep}_{B,D}(id) =\begin{pmatrix}
   -1 & 2  & 0\\
    2 & 3  & -1\\
    0 & 1  & 4
\end{pmatrix}$.  (This problem is very easy!)
\end{problem}

\begin{answer}
% Answer here for problem number 1,  points 3
\end{answer}




\begin{problem}
Consider the bases $B$ and $D$ for $\R^3$, as given here:
$$
   B = \left\langle
          \begin{pmatrix} 1\\ 2\\ 3 \end{pmatrix},
          \begin{pmatrix} 0\\ -3\\ 2 \end{pmatrix},
          \begin{pmatrix} -2\\ 4\\ -3 \end{pmatrix}
       \right\rangle
   \qquad\qquad\qquad
   D = \left\langle
          \begin{pmatrix} 1\\ 0\\ 1 \end{pmatrix},
          \begin{pmatrix} 0\\ 1\\ -1 \end{pmatrix},
          \begin{pmatrix} 0\\ -1\\ 0 \end{pmatrix}
       \right\rangle
$$
\begin{itemize}
\item[\bf(a)] Find the change of basis matrix ${\rm Rep}_{B,D}(id)$.
\item[\bf(b)] Check that ${\rm Rep}_{B}\begin{pmatrix} -1\\0\\4 \end{pmatrix} = 
                                       \begin{pmatrix} 1\\2\\1 \end{pmatrix} $.
              (You're not asked to find the representation, just check it.)
\item[\bf(c)] The change of basis matrix must satisfy  
${\rm Rep}_{B,D}(id)\cdot {\rm Rep}_B(\vec v) = {\rm Rep}_D(\vec v)$.
Verify that in fact $${\rm Rep}_{B,D}(id)\cdot {\rm Rep}_B\begin{pmatrix} -1\\0\\4 \end{pmatrix} 
= {\rm Rep}_D\begin{pmatrix} -1\\0\\4 \end{pmatrix}$$
\end{itemize}
\end{problem}

\begin{answer}
% Answer here for problem number 2,  points 5
\end{answer}




\begin{problem}Find an affine transformation $f\colon\R^2\to\R^2$ such that
$$
   f\begin{pmatrix} 0\\ 0 \end{pmatrix} = \begin{pmatrix} -2\\ 3 \end{pmatrix}\qquad\qquad
   f\begin{pmatrix} 1\\ 0 \end{pmatrix} = \begin{pmatrix} 1\\ 1 \end{pmatrix}\qquad\qquad
   f\begin{pmatrix} 0\\ 1 \end{pmatrix} = \begin{pmatrix} 2\\ 5 \end{pmatrix}
$$
Hint:  It's easy to find the translation\vadjust{\smallskip} part of the affine map!
Recall that $f$ can be represented in the form $f\begin{pmatrix} x\\ y \end{pmatrix} 
    = \begin{pmatrix} ax+by+e\\ cx+dy+f \end{pmatrix}
    = \begin{pmatrix} a& b\\ c&d \end{pmatrix}
          \begin{pmatrix} x\\ y \end{pmatrix} +
          \begin{pmatrix} e\\ f \end{pmatrix}$.
\end{problem}

\begin{answer}
% Answer here for problem number 3,  points 3
\end{answer}




\begin{problem}
The cross product of two vectors $\vec v,\vec w\in \R^3$ is a vector, $\vec v\times \vec w$, that 
is orthogonal  to both $\vec v$ and $\vec w$.  The cross product can be computed as the
``formal determinant''

\medskip
\centerline{$\begin{pmatrix}a_1\\a_2\\a_3\end{pmatrix}
            \times \begin{pmatrix}b_1\\b_2\\b_3\end{pmatrix}
            =\begin{vmatrix}\vec e_1&\vec e_2&\vec e_3\\ a_1&a_2&a_3\\ b_1&b_2&b_3\end{vmatrix}$}

\medskip
\noindent Find $\begin{pmatrix} 3\\-1\\2 \end{pmatrix} \times \begin{pmatrix} 1 \\ 4 \\ -2 \end{pmatrix}$
by writing out the formal determinant 
$\begin{vmatrix}\vec e_1&\vec e_2&\vec e_3\\ 3& -1 & 2 \\ 1 & 4 & -2 \end{vmatrix}$,
using the formula\vadjust{\medskip} for the determinant of a $3\times3$ matrix,
and show that the result is, in fact, orthogonal to both vectors.
\end{problem}

\begin{answer}
% Answer here for problem number 4,  points 3 
\end{answer}




\begin{problem}
We looked at a formula for computing the determinant of a $3\times3$ matrix.  That formula can be derived using
Laplace's expansion for the determinant.  Apply Laplace's expansion to the general $3\times3$ determinant
$$\begin{vmatrix} a_1& a_2& a_3 \\ b_1& b_2& b_3 \\ c_1& c_2& c_3 \end{vmatrix}$$
to derive the formula for the determinant of a $3\times3$ matrix.  (You only need to apply Laplace's expansion
for the first step of the computation, not for the resulting $2\times2$ matrices.)
\end{problem}

\begin{answer}
% Answer here for problem number 5,  points 2
\end{answer}




\begin{problem}
Compute each of the following determinants.  Some of them are very easy, using properties of
the determinant, and you should check for that before doing a complex computation.
For any problem where you use a property of the determinant to find the
answer, state the property that you use.

\bigskip
\centerline{
   \textbf{(a)}\quad $\begin{vmatrix} 3 & 5 \\ 2 & -1 \end{vmatrix}$ \hskip 0.7 in
   \textbf{(b)}\quad $\begin{vmatrix} 5 & 7 & 4 \\ 0 & 2 & 3 \\ 0 & 0 & 3 \end{vmatrix}$ \hskip 0.7 in
   \textbf{(c)}\quad $\begin{vmatrix} 1 & 3 & -2 \\ 2 & 1 & 1 \\ 3 & 2 & 4 \end{vmatrix}$
}

\bigskip
\centerline{
   \textbf{(d)}\quad $\begin{vmatrix} 0 & 0 & 3 \\  0 & 2 & -3 \\ 4 & 5 & -1 \end{vmatrix}$ \hskip 0.5 in
   \textbf{(e)}\quad $\begin{vmatrix} 1 & 2 & 4 & -1 \\ 3 & 5 & -3 & 7 \\
                                      1 & 2 & 4 & -1 \\ 6 & 2 & 3 & 7 \end{vmatrix}$ \hskip 0.5 in
   \textbf{(f)}\quad $\begin{vmatrix} 1 & 2 & 3 & -1 \\ 0 & -1 & 2 & 3 \\
                                      2 & 5 & 6 & 1 \\ -1 & 1& 1 & 3 \end{vmatrix}$
}

\end{problem}

\begin{answer}
% Answer here for problem number 6,  points 9
\end{answer}





\end{document}

