\documentclass[12pt]{article}
\usepackage{amsmath}
\usepackage{amsfonts}
\usepackage{amssymb}
\usepackage[mathscr]{eucal}
\usepackage{mathrsfs}
\usepackage{graphicx}

\def\scaledgraphics#1#2{\includegraphics[width=#1]{#2}}

%\setlength{\topmargin}{-0.3 true in}
%\setlength{\headsep}{0 true in}
%\setlength{\topskip}{0 true in}
%\setlength{\textwidth}{6.5 true in}
%\setlength{\oddsidemargin}{0 true in}
%\setlength{\evensidemargin}{0 true in}
%\setlength{\textheight}{8.5 true in}
%\pagestyle{empty}
%\renewcommand{\arraystretch}{1.2}

\setlength{\topmargin}{-0.1 true in}
\setlength{\headsep}{0 true in}
\setlength{\topskip}{0 true in}
\setlength{\textwidth}{6.5 true in}
\setlength{\oddsidemargin}{0 true in}
\setlength{\evensidemargin}{0 true in}
\setlength{\textheight}{8.2 true in}
\pagestyle{empty}



\newcounter{problemcounter}
\newcounter{partcounter}[problemcounter]

\newcommand{\Item}[1]{\par\hangafter=0
                         \hangindent=15pt
                         \noindent\llap{#1}\ignorespaces}
\newcommand{\IItem}[1]{\par\hangafter=0
                         \hangindent=40pt
                         \noindent\llap{#1}\ignorespaces}
                         
\newcommand{\stack}[1]{\vtop{\halign{##\hfil\cr #1\crcr}}}

\newcommand{\problem}{\bigskip\bigskip\stepcounter{problemcounter}\Item{\bfseries\arabic{problemcounter}.\ }}
\newcommand{\ppart}{\bigskip\medskip\stepcounter{partcounter}\IItem{\bfseries\alph{partcounter})\ }}
\newcommand{\pparts}[1]{\medskip
   \halign{\hskip40pt\stepcounter{partcounter}\llap{\bfseries\alph{partcounter})\ }$##$\qquad\hfil&&
            \hskip30pt\stepcounter{partcounter}\llap{\bfseries\alph{partcounter})\ }$##$\qquad\hfil\cr
   #1\crcr}}
%\dspparts is like \pparts, but uses \displaystyle
\newcommand{\dspparts}[1]{\medskip
   \halign{\hskip40pt\stepcounter{partcounter}\llap{\bfseries\alph{partcounter})\ }$\displaystyle ##$\qquad\hfil&&
            \hskip30pt\stepcounter{partcounter}\llap{\bfseries\alph{partcounter})\ }$\displaystyle ##$\qquad\hfil\cr
   #1\crcr}}
%\tparts is same as \pparts, but doesn't use math mode
\newcommand{\tparts}[1]{\medskip
   \halign{\hskip40pt\stepcounter{partcounter}\llap{\bfseries\alph{partcounter})\ }##\qquad\hfil&&
            \hskip30pt\stepcounter{partcounter}\llap{\bfseries\alph{partcounter})\ }##\qquad\hfil\cr
   #1\crcr}}

\newcommand{\N}{{\mathbb N}}
\newcommand{\Z}{{\mathbb Z}}
\newcommand{\Q}{{\mathbb Q}}
\newcommand{\R}{{\mathbb R}}
\newcommand{\C}{{\mathbb C}}
\newcommand{\Zpos}{{{\mathbb Z}^+}}
\newcommand{\SUB}{\subseteq}
\newcommand{\SUP}{\supseteq}
\newcommand{\PSUB}{\varsubsetneq}
\newcommand{\PSUP}{\varsupsetneq}
\newcommand{\st}{\,|\,}

\let\ds=\displaystyle

\pretolerance 1000


\begin{document}


\hbox to \hsize{\bf Math 204, Fall 2020\hfil Final Exam}
\nointerlineskip
\vskip 2 pt
\hrule
\bigskip

\centerline{\it Your work should be submitted before the official final exam time,}
\centerline{\it 8:30 AM, Wednesday, December 9.}

\bigskip

{\narrower

\textbf{About the exam:} Counting two tests, the final exam, and counting the homework grade
twice, you will have five grades for this course.  The lowest of those grades will be dropped.
You can choose not to take the final exam; in that case, you will keep the grade that you
have before the exam.  After the final homework assignment is graded, I will let you know
what that grade would be.  Note that you cannot lower your grade by taking the final exam.

This is a takehome exam, but with a short Zoom meeting to discuss your responses.
As usual for this course, you can submit your work through Canvas or, if you choose to
do it, as an overleaf.com project.  Your work is due by the officially scheduled final
exam period, 8:30~AM on Wednesday, December~9.

You can make an appointment to meet with me during the final exam period through the
Canvas calendar, in the same way that you would for office hours.  The Zoom meeting
will use the same link as the one used for office hours, which can be found on the
Canvas page for this course.  If you cannot meet during the regular exam period for
some reason, you should email me to set up an alternative time.

The work that you submit for this exam should be your own.  You can use
course materials, including the textbook, your notes, videos from online lectures,
and posted solutions to homeworks.  You should not use other books or material from
the Internet.   You can ask me questions about the
exam, but you should not receive help on the exam from
other students, your friends and family, or anyone else.

For the problems on this exam,  you should present neatly written solutions.
Write out your answers carefully, including explanations to justify your work when appropriate.

For the essay questions on this exam, you should write out clear and well-organized responses
in complete sentences and paragraphs.  The questions are meant to give you an opportunity to
display your understanding of the central ideas from the course.
Please type your answers to essay questions in LaTeX or in a word processing application.
You can submit your essays as part of an overleaf.com project or through Canvas as separate 
documents.  If you are submitting your work through Canvas, you must save your essays as a PDF 
file or as a Microsoft Word document.

If you are confused about any of the problems on the test, or if you have other
questions, you can email me.  I will also have some office hour times available in the
Canvas calendar before the exam is due. While I won't give extensive help, I might be able
to give some hints.  

}

\bigskip
\hrule

\vfil\eject

\medskip

\def\pm#1{\begin{pmatrix} #1 \end{pmatrix}}
\def\vm#1{\begin{vmatrix} #1 \end{vmatrix}}

\def\equations#1{
   \vtop{
      \halign{\hfil$##\ $&&\hfil$\ ##\ $\hfil&\hfil$\ ##\ $\cr #1\crcr}
   }
}


\problem \textbf{Calculations.} (35 points) Remember to show all work and provided explanations.
For example, if you are asked whether something is a basis, you need to explain \textit{why}
your work shows that it is or is not a basis.  Some of the calculations are longer than others,
but each one counts for 7 points.  (Part \textbf{e} is the hardest.)

\medskip

\ppart Find all solutions of the homogeneous system:\ \ 
\equations{
    x &+&  2y   &&   &-& w &=& 0\cr
    2x &+&   5y   &-&  6z &+&  2w &=& 0\cr
     &&    2y   &+&  2z &+&  w &=& 0\cr
    -2x &-&    y   &+&  4z &+&  3w &=& 0\cr
}

\ppart Is $\left\langle \pm{1\\3\\-2}, \pm{4\\7\\-3}, \pm{3\\-1\\4} \right\rangle$ a basis of $R^3$?

\ppart Let ${\mathscr P}_3$ be the vector space of polynomials of degree less than or equal to 3.
Let $B=\langle x^3-x^2,x^2-x,x-1,1 \rangle$, which is a basis of ${\mathscr P}_3$.  
Find $Rep_B(4x^3+3x^2+2x+1)$.

\ppart Find the determinant:\ \ 
$\vm{ 0&3&0&0\\ 1&4&2&5\\ 3&3&3&0\\ 5&2&6&0 }$

\ppart Find all eigenvalues of the following matrix, and find an eigenvector of the
matrix for each eigenvalue.  Two of the eigenvalues are complex.  (Hint: When looking for
the eigenvectors, you can take the third component of the eigenvector to be 1.)
$$\begin{pmatrix}
 3 & 0 & 0 \\
 1 & 2 & -1 \\
 -2 & 1 & 2 
\end{pmatrix}$$

\bigskip
\hrule
\medskip

\problem \textbf{Proofs.} (25 points)  Remember to write clear and rigorous proofs, and to be clear about
the justifications for your assertions.  Each part counts for 6 points, except for part \textbf{a}, which counts for 7 points.

\medskip

\ppart The set $M_{2\times 2}$ of $2\times 2$ real matrices is a vector space over $\R$.
Let $W$ be the subset of $M_{2\times 2}$ given by $W=\big\{\big(\begin{smallmatrix} a & b \\ -b & a\end{smallmatrix}
\big)\,\,\big|\,\,a,b\in\R\big\}$.
Show that $W$ is a subspace of $M_{2\times 2}$, and find a basis for $W$.


\ppart Let $V,W$ be a vector spaces.  Let $f\colon V\to W$ and $g\colon V\to W$ be homomorphisms.  Define $h\colon V\to W$
by $h(\vec v) = f(\vec v) + g(\vec v)$.  Show that $h$ is a homomorphism.


\ppart Let $\langle\vec v_1,\vec v_2,\dots,\vec v_n\rangle$ be a basis of the real vector space $\R^n$.  So, $\vec v_j\in\R^n$ for each $j$.
But we can also consider $\vec v_j$ to be in the complex vector space $\C^n$.  Show that 
$\langle\vec v_1,\vec v_2,\dots,\vec v_n\rangle$ is a basis for the complex vector space $\C^n$.
(Hint:  Any $\vec w\in\C^n$ can be written in the form $\vec w = \vec a + \vec b\,i$, where $\vec a,\vec b\in\R^n$.)


\ppart Suppose that $A$ is an $n\times n$ real matrix that has $n$ \textbf{distinct} real eigenvalues $\lambda_1,\lambda_2,\dots,\lambda_n$.
Show that if $|\lambda_i|< 1$ for $i=1,2,\dots,n$, then the determinant satisfies $|\det(A)|< 1$.  (Note that the notation $|\cdot|$ 
in this problem represents absolute value, not determinant.) 


\bigskip
\bigskip
\hrule
\bigskip

\problem \textbf{Essay Questions.} (40 points) Remember to type your answers and to submit them as
a PDF file, a Microsoft Word document, or as part of an overleaf.com project.  Remember to write your responses
as well-organized essays in full sentences and paragraphs. Each question counts for 20 points.  Your answer to
each essay question should be at least one full page, but should not be longer than two pages (double-spaced).  You do not have
to deal with complex numbers or with complex vector spaces in these essays.

\ppart We began the course with \textit{linear systems} and how to solve them using \textit{row reduction}.
Discuss some of the ways these concepts have been useful in other parts of the course, such as working
with linear independence, matrix inverses, determinants, and eigenvalues.

\ppart A large part of the course was about matrices and the specific vector spaces $\R^n$.  However, we also worked
with other more general vector spaces, $V$.  In many cases, we worked with general finite-dimensional vector spaces by
choosing bases for those vector spaces.  Explain what it means to ``choose a basis'' and discuss how doing so
lets us study general vector spaces.  (How does a choice of basis relate a vector space $V$ to $\R^n$?  What about
homomorphisms, determinants, eigenvalues, and so on?)


\end{document}



