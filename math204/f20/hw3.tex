\documentclass[12pt]{article}
\usepackage[utf8]{inputenc}
\usepackage[letterpaper]{geometry}
\usepackage{amsmath}
\usepackage{amsfonts}
\usepackage{amsthm}
\usepackage{amssymb}
\usepackage{color}
\setlength{\topmargin}{-0.3 true in}
\setlength{\headsep}{0 true in}
\setlength{\topskip}{0 true in}
\setlength{\textwidth}{6.5 true in}
\setlength{\oddsidemargin}{0 true in}
\setlength{\evensidemargin}{0 true in}
\setlength{\textheight}{9 true in}
\renewcommand{\arraystretch}{1.2}
\newcommand{\eps}{\varepsilon}
\newcommand{\R}{{\mathbb R}}
\newcommand{\N}{{\mathbb N}}
\newcommand{\Q}{{\mathbb Q}}
\newcommand{\Z}{{\mathbb Z}}
\theoremstyle{definition}
\newtheorem{problem}{Problem}
\definecolor{darkblue}{RGB}{0,0,150}
\definecolor{darkred}{RGB}{180,0,0}
\definecolor{darkgreen}{RGB}{0,120,0}
\newenvironment{answer}{\par\bigskip\bgroup\color{darkblue}}{\egroup}
\newenvironment{grading}{\bgroup\color{darkred}}{\egroup}
\newenvironment{secondans}{\bgroup\color{darkgreen}}{\egroup}
\pagestyle{empty}

\def\rowop#1{\qquad\xrightarrow{\begin{matrix}#1\end{matrix}}\qquad}

\begin{document}


\hbox to \hsize{\bf Math 204, Fall 2020\hfil Homework \#3}
\nointerlineskip
\vskip 2pt
\hrule height 1pt

\medskip

\centerline{\textit{This homework is due by the end of the day on Wednesday, Sept. 16.}}
\centerline{\textit{Be sure to show your work and explain your reasoning!}}

\medskip

\begin{problem}
Produce \textbf{three} other matrices that are row equivalent to the following matrix. 
State what you did to get each new matrix.   If you understand what row equivalence
means, this exercise is very easy!
$$\begin{pmatrix}
    3&4&-2\\
    1&1&7\\
    -5&3&1
\end{pmatrix}$$
\end{problem}

\begin{answer}
%You can put your answer here
\end{answer}


\begin{problem}
Apply row operations to put the matrix on the left below into reduced echelon form.  Show the full sequence
of operations that you apply.  The only correct answer is shown on the right.  The point of the problem is
to show the computation.

\bigskip
\centerline{
 $\begin{pmatrix}
    1 & 0 & -2 & 0 & 2\\
    0 & 1 &  3 & 2 & 7\\
    1 & 2 &  4 & 2 & 8\\
   -2 & 2 & 10 & 2 &2
 \end{pmatrix}$
\qquad\qquad\qquad
 $\begin{pmatrix}
    1 & 0 & -2 & 0 & 2\\
    0 & 1 &  3 & 0 & -1\\
    0 & 0 &  0 & 1 & 4\\
    0 & 0 &  0 & 0 & 0
 \end{pmatrix}$
}

\end{problem}


\begin{answer}
%You can put your answer here
\end{answer}




\begin{problem} 
Assuming that the matrix in the previous problem represents a \textbf{homogeneous} system of linear equations,
use the reduced echelon form of the matrix to find the solution set of that system.  (The constant terms in the
equations, which are all zero, are omitted from the matrix!)
\end{problem}


\begin{answer}
%You can put your answer here
\end{answer}


\begin{problem}
Put the following eight matrices into groups, so that each matrix is row-equivalent to all the other
matrices in the group, but not row-equivalent to matrices from other groups.  (Remember how to use
reduced echelon form to tell whether two matrices are row equivalent.)

\bigskip
\centerline{
   \textbf{a)} $\begin{pmatrix} -1& 3 \\ 2 &-6\end{pmatrix}$ \hskip 0.3in
   \textbf{b)} $\begin{pmatrix} 2 & 4 \\ 1 & 5 \end{pmatrix}$ \hskip 0.3in
   \textbf{c)} $\begin{pmatrix} 0 & 1 \\ 0 & 3 \end{pmatrix}$ \hskip 0.3in
   \textbf{d)} $\begin{pmatrix} 3 & 6 \\ -2 & -4 \end{pmatrix}$ \hskip 0.3in
}

\bigskip
\centerline{
   \textbf{e)} $\begin{pmatrix} 1 & 1 \\ 1 & -1 \end{pmatrix}$ \hskip 0.3in
   \textbf{f)} $\begin{pmatrix} \frac{1}{3} & -1 \\ -3 & 9\end{pmatrix}$ \hskip 0.3in
   \textbf{g)} $\begin{pmatrix} 0 & 0 \\ 1 & 2 \end{pmatrix}$ \hskip 0.3in
   \textbf{h)} $\begin{pmatrix} 0 & 2 \\ 3 & 7 \end{pmatrix}$ \hskip 0.3in
}

\end{problem}


\begin{answer}
%You can put your answer here
\end{answer}



\begin{problem} 
Let $W$ be a subset of $\R^2$ that is a vector space using the addition and scalar multiplication operations from $\R^2$.
Suppose that $\begin{pmatrix} 1\\ 0 \end{pmatrix}\in W$ and $\begin{pmatrix} -2\\ 1\end{pmatrix}\in W$.  
Prove that $W$ is all of $\R^2$.  (This is easy, as long as you remember that any vector space is closed under 
vector addition and scalar multiplication.  This means that, given any vector is $\R^2$, you just need to be
able to write that vector as a linear combination of the two given vectors.)
\end{problem}


\begin{answer}
%You can put your answer here
\end{answer}



\begin{problem}
Remember that to show that a set is \textbf{not} a vector space, you only need to find one vector space property that fails,
out of the ten properties that vector spaces must satisfy.
\begin{enumerate}
\item[(a)] Let $S$ be the subset of $\R^2$ defined as $S=\{(x,y)\,|\,x+y=1\}$. Show that $S$, using
the addition and scalar multiplication operations from $\R^2$, is \textbf{not} a vector space.  
\item[(b)] Let $T$ be the subset of $\R^2$ defined as $T=\{(x,y)\,|\,x+y\ge 0\}$. Show that $T$, using
the addition and scalar multiplication operations from $\R^2$, is \textbf{not} a vector space.
\end{enumerate}
\end{problem}


\begin{answer}
%You can put your answer here
\end{answer}



\end{document}



