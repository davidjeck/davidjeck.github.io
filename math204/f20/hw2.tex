\documentclass[12pt]{article}
\usepackage[utf8]{inputenc}
\usepackage[letterpaper]{geometry}
\usepackage{amsmath}
\usepackage{amsfonts}
\usepackage{amsthm}
\usepackage{amssymb}
\usepackage{color}
\setlength{\topmargin}{-0.3 true in}
\setlength{\headsep}{0 true in}
\setlength{\topskip}{0 true in}
\setlength{\textwidth}{6.5 true in}
\setlength{\oddsidemargin}{0 true in}
\setlength{\evensidemargin}{0 true in}
\setlength{\textheight}{9 true in}
\renewcommand{\arraystretch}{1.2}
\newcommand{\eps}{\varepsilon}
\newcommand{\R}{{\mathbb R}}
\newcommand{\N}{{\mathbb N}}
\newcommand{\Q}{{\mathbb Q}}
\newcommand{\Z}{{\mathbb Z}}
\theoremstyle{definition}
\newtheorem{problem}{Problem}
\definecolor{darkblue}{RGB}{0,0,150}
\definecolor{darkred}{RGB}{180,0,0}
\definecolor{darkgreen}{RGB}{0,120,0}
\newenvironment{answer}{\bgroup\color{darkblue}}{\egroup}
\newenvironment{grading}{\bgroup\color{darkred}}{\egroup}
\newenvironment{secondans}{\bgroup\color{darkgreen}}{\egroup}
\pagestyle{empty}

\def\rowop#1{\qquad\xrightarrow{\begin{matrix}#1\end{matrix}}\qquad}

\begin{document}


\hbox to \hsize{\bf Math 204, Fall 2020\hfil Homework \#2}
\nointerlineskip
\vskip 2pt
\hrule height 1pt

\medskip

\centerline{\textit{This homework is due by the end of the day on Wednesday, Sept. 9.}}
\centerline{\textit{Be sure to show your work and explain your reasoning!}}

\medskip

\begin{problem}
Two people solve a linear system of equations in two variables and they get the following sollution sets, where
each set represents a line in $\R^2$:

\medskip

\centerline{
  $A=\left\{ \begin{pmatrix} 3\\2 \end{pmatrix} + a \cdot \begin{pmatrix} 1\\3 \end{pmatrix} \ :\ a\in\R\right\}$
  \qquad
  $B=\left\{ \begin{pmatrix} 1\\-4 \end{pmatrix} + a \cdot \begin{pmatrix} 2\\6 \end{pmatrix} \ :\ a\in\R\right\}$
}

\medskip

\noindent Can they both be correct?  Explain why the two lines are actually the same line.  First check that the point
$(3,2)$ is on both lines.  Then explain why the two lines point in the same direction.  And explain in words why
all this shows that the two lines are the same.
\end{problem}

\begin{answer}
%Erase this line and put your answer here!
\end{answer}



\begin{problem}
Suppose two planes in $\R^3$ are given by the linear equations $x+y+z=1$ and $Ax+By+Cz=D$.
The intersection of the two planes can be empty, or it can be a line, or the planes could be
identical.  For each case, what has to be true about the constants  $A$, $B$, $C$, and $D$ in 
the second equation? Explain!
(Hint: The intersection is the set of solutions to a system of two linear equations, and
that set can be determined by putting the system into echelon form.)
\end{problem}

\begin{answer}
%Erase this line and put your answer here!
\end{answer}



\begin{problem}
Let $\vec v_1, \vec v_2, \dots, \vec v_{n-1}$ be $n-1$ vectors in $\R^n$.  Prove that there
is a non-zero vector $\vec x$ in $\R^n$ that is orthogonal to $\vec v_i$ for all $i=1,2,\dots,n-1$.
(Hint: Think about linear equations!  Write the condition as a linear system, and note that
it is a homogeneous system.)
\end{problem}

\begin{answer}
%Erase this line and put your answer here!
\end{answer}



\begin{problem}
Let $\vec v_1=\begin{pmatrix} 1\\2\\-1 \end{pmatrix}$,
    $\vec v_2=\begin{pmatrix} 3\\0\\1 \end{pmatrix}$, and
    $\vec v_3=\begin{pmatrix} 1\\1\\2 \end{pmatrix}$.  
    Write the vector $\begin{pmatrix} 0\\5\\-1 \end{pmatrix}$
as a linear combination of $\vec v_1$, $\vec v_2$, and $\vec v_3$. To find the coefficients
in the linear combination, set up a system of linear equations, and then solve that system.
\end{problem}

\begin{answer}
%Erase this line and put your answer here!
\end{answer}


\problem Apply Gauss's method to put each matrix into echelon form.
Based on your answer, state whether the matrix is singular or non-singular. 

\medskip

\textbf{(a)} 
   $\begin{pmatrix}
      1& 1\\
      1& 2
   \end{pmatrix}$
\qquad\qquad\textbf{(b)}
   $\begin{pmatrix}
      -1& 0& 1& 0\\
      3& 2& -2&4\\
      2& 1& 0& -1\\
      1& 1&0 &1 
   \end{pmatrix}$


\begin{answer}
%Erase this line and put your answer here!
\end{answer}



\end{document}



