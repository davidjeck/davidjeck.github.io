\documentclass[11pt]{article}
\usepackage[utf8]{inputenc}
\usepackage[letterpaper]{geometry}
\usepackage{amsmath}
\usepackage{amsfonts}
\usepackage{amsthm}
\usepackage{amssymb}
\usepackage{mathrsfs}
\usepackage{color}
\usepackage{graphicx}
\def\scaledgraphics#1#2{\includegraphics[width=#1]{#2}}
\setlength{\topmargin}{-0.4 true in}
\setlength{\headsep}{0 true in}
\setlength{\topskip}{0 true in}
\setlength{\textwidth}{6.5 true in}
\setlength{\oddsidemargin}{0 true in}
\setlength{\evensidemargin}{0 true in}
\setlength{\textheight}{9.5 true in}
\renewcommand{\arraystretch}{1.2}
\newcommand{\eps}{\varepsilon}
\newcommand{\R}{{\mathbb R}}
\newcommand{\N}{{\mathbb N}}
\newcommand{\Q}{{\mathbb Q}}
\newcommand{\Z}{{\mathbb Z}}
\newcommand{\C}{{\mathbb C}}
\theoremstyle{definition}
\newtheorem{problem}{Problem}
\definecolor{darkblue}{RGB}{0,0,150}
\definecolor{darkred}{RGB}{180,0,0}
\definecolor{darkgreen}{RGB}{0,120,0}
\newenvironment{answer}{\par\bigskip\bgroup\color{darkblue}}{\egroup}
\newenvironment{grading}{\par\medskip\bgroup\color{darkred}}{\egroup\par\bigskip}
\newenvironment{secondans}{\bgroup\color{darkgreen}}{\egroup}
\pagestyle{empty}


\def\rowop#1{\qquad\xrightarrow{\begin{matrix}#1\end{matrix}}\qquad}

\begin{document}


\hbox to \hsize{\bf Math 204, Fall 2020\hfil  Homework \#10}
\nointerlineskip
\vskip 2pt
\hrule height 1pt

\medskip

\centerline{\textit{This homework is due by 11:59 PM on Tuesday, December 1}}

\medskip



\begin{problem} %problem 1
A few problems on complex numbers\dots
\begin{itemize}
\item[\bf(a)] 
   Recall that for a complex number $z=a+bi$, $|z|$ is defined to be
   $|z|=\sqrt{a^2+b^2}$.  Verify that for two complex numbers $z$ and
   $w$, $|zw| = |z|\cdot|w|$.
\item[\bf(b)] 
   Suppose that $w_0=a+bi$ is some complex number.  Recall that the
   conjugate of $w_0$ is defines as $\overline{w_0}=a-bi$.  
   Let $p(z)$ be the polynomial $p(z) = (z-w_0)(z-\overline{w_0})$.
   Verify that when $p(z)$ is written in standard form as
   $p(z) = c_0 + c_1z + c_2z^2$, all of the coefficients $c_0$,
   $c_1$, and $c_2$ are real.
\item[\bf(c)] 
   Use the identity $e^{i\theta}=\cos(\theta) + i \sin(\theta)$ and the fact that
   $(e^{i\theta})^2 = e^{2i\theta}$ to derive the usual double angle formulas:
   $\cos(2\theta)=\cos^2(\theta)-\sin^2(\theta)$ and 
   $\sin(2\theta)=2\sin(\theta)\cos(\theta)$. 
\end{itemize}
\end{problem}

\begin{answer}
% Answer here for problem number 1, 5 points 
\end{answer}



\begin{problem} %problem 2
Find all the eigenvalues, real or complex, of the following matrices.  (Note that
one of these is really easy.)

\bigskip
\centerline{
  \textbf{(a)}\ \ $\begin{pmatrix} 1 & 2 \\ -2 & 1  \end{pmatrix}$\qquad\qquad
  \textbf{(b)}\ \ $\begin{pmatrix} 3+i & 0 & 5\\ 0 & 2 & 7\\ 0 & 0 & -i  \end{pmatrix}$\qquad\qquad
  \textbf{(c)}\ \ $\begin{pmatrix}  5 & 0 & 0 \\ -2 & 3 & 6\\ 0 & 1 & -2  \end{pmatrix}$
}
\end{problem}

\begin{answer}
% Answer here for problem number 2,  points 5
\end{answer}



\begin{problem} %problem 3
Suppose that $A$ is an $n\times n$ matrix, and $\lambda$ is an eigenvalue for $A$.
Show that $\lambda^2$ is an eigenvalue for $AA$.  [Hint:  Let $\vec v$ be an
eigenvector for $A$ with eigenvalue $\lambda$.]
\end{problem}

\begin{answer}
% Answer here for problem number 3,  points 3
\end{answer}



\begin{problem} %problem 4
Let $h\colon \C^2\to\C^2$ be a homomorphism that has eigenvalues $1-i$ and $1+i$.
Suppose that $\begin{pmatrix} 1\\ -2 \end{pmatrix}$ is an eigenvector with eigenvalue $1-i$,
and that $\begin{pmatrix} 1\\ 1\end{pmatrix}$ is an eigenvector with eigenvalue $1+i$.
Find the matrix for $h$ in the standard basis, $\langle\vec e_i,\vec e_2\rangle$.
[Hint:  You need to find $h(\vec e_1)$ and $h(\vec e_2)$.]
\end{problem}

\begin{answer}
% Answer here for problem number 4,  points 3
\end{answer}



\begin{problem} %problem 5
Let ${\mathscr D}$ be the vector space of differentiable functions
from $\R$ to $\R$.  That is, ${\mathscr D}$ is the set $\{f\colon \R\to\R\,|\, f'(x)\text{ exists for all }x\}$,
with the usual addition and scalar multiplication for real-valued functions.
Let $\partial\colon{\mathscr D}\to{\mathscr D}$ be the derivative function,
$\partial(f)=f'$.  Show that every real number $\lambda$ is an eigenvalue for $\partial$,
and find an eigenvector for eigenvalue $\lambda$.  [Hint: What is the derivative of $e^{ax}$? Once you
remember that derivative, this question is trivial.]

\end{problem}

\begin{answer}
% Answer here for problem number 5,  points 2
\end{answer}







\end{document}

