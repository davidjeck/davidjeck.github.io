\documentclass[11pt]{article}
\usepackage[utf8]{inputenc}
\usepackage[letterpaper]{geometry}
\usepackage{amsmath}
\usepackage{amsfonts}
\usepackage{amsthm}
\usepackage{amssymb}
\usepackage{color}
\setlength{\topmargin}{-0.3 true in}
\setlength{\headsep}{0 true in}
\setlength{\topskip}{0 true in}
\setlength{\textwidth}{6.5 true in}
\setlength{\oddsidemargin}{0 true in}
\setlength{\evensidemargin}{0 true in}
\setlength{\textheight}{9 true in}
\renewcommand{\arraystretch}{1.2}
\newcommand{\eps}{\varepsilon}
\newcommand{\R}{{\mathbb R}}
\newcommand{\N}{{\mathbb N}}
\newcommand{\Q}{{\mathbb Q}}
\newcommand{\Z}{{\mathbb Z}}
\theoremstyle{definition}
\newtheorem{problem}{Problem}
\newenvironment{answer}{\bgroup\color{blue}}{\egroup}
\newenvironment{grading}{\bgroup\color{red}}{\egroup}
\newenvironment{secondans}{\bgroup\color{green}}{\egroup}


\def\rowop#1{\qquad\xrightarrow{\begin{matrix}#1\end{matrix}}\qquad}

\begin{document}


\hbox to \hsize{\bf Math 204, Fall 2020\hfil Homework \#1}
\nointerlineskip
\vskip 2pt
\hrule height 1pt

\medskip

\centerline{\textit{This homework on Sections 1.I.1 and 1.I.2 is due by the end of the day on Monday, Aug. 31.}}
\centerline{\textit{Be sure to show your work and explain your reasoning!}}

\medskip
\hrule
\medskip

Note: Here is an example of applying row operations to a linear systtem:
\begin{align*}
    \begin{matrix}
       x  &+ 3y &- 2z &= &1\\
     -2x  &+ y  &- z &= &-3\\
       x  &- y  &+ z &= &2 
    \end{matrix}
  & \rowop{2\rho_1 + \rho_2 \\ -\rho_1 +\rho_3}
    \begin{matrix}
       x &+ 3y &- 2z &= &1\\
         &  7y &- 5z &= &-1\\
         &- 4y &+ 3z &= &1 
    \end{matrix}
\\[15pt]
  & \rowop{\frac{4}{7}\rho_4 + \rho_3}
    \begin{matrix}
       x &+ 3y &- 2z &= &1\\
         &  7y &- 5z &= &-1\\
         &     & \frac{1}{7}z &= &\frac{3}{7} 
    \end{matrix}
\end{align*}
 
\hrule
\bigskip


\begin{problem}
The following systems of linear equations have unique solutions.  Use 
Gauss's Method to  put each system to echelon form, and then find the solution.
As you apply row operations, show the result of each operation and 
show which operation you are applying.  You can specify
a row operation using the same notation as the textbook.
You can combine several row operations of the form $\rho_j+k\rho_i$ 
into one step, as long as they use the same $\rho_i$.

\bigskip

\textbf{(a)} \qquad
    $\begin{matrix}
       2x & -3y & = & -1 \\
       x  & +2y & = & 3
    \end{matrix}$

\bigskip\bigskip

\textbf{(b)} \qquad
    $\begin{matrix}
           &   x_2 & +2x_3 & = & 3 \\
       x_1 &  -x_2 & -3x_3 & = & -2 \\
      2x_1 & +4x_2 &  -x_3 & = & 0
    \end{matrix}$
    
\bigskip\bigskip
    
\textbf{(c)} \qquad
    $\begin{matrix}
       x &  +y &  +z & = & 1\\
       x &  -y & -2z & = & 2\\
      2x &  +y &  +z & = & 3\\
       x &  -y &     & = & 4
    \end{matrix}$

\end{problem}
\begin{answer}
% Erase this line and put your answer here!
\end{answer}

\bigskip

\begin{problem}
For each of the linear systems in problem {\bf 1}, rewrite the system in the form
of an augmented matrix.  For this short problem, {\it you do not need to show any work, just
write the answers}.  You just have to write the augmented matrix form of the original
system of equations.
\end{problem}
\begin{answer}
% Erase this line and put your answer here!
\end{answer}

\bigskip

\begin{problem} 
The following systems are already in echelon form.  Each each system has an 
infinite number of solutions.  Express the set of solutions in vector form.  The answers
will have a form similar to $\{\vec v_1 + a\vec v_2 + b\vec v_3 \,|\, a,b\in\R\}$,
where $v_1$, $v_2$ and $v_3$ are column vectors of constants.

\bigskip

\textbf{(a)} \qquad
    $\begin{matrix}
       x & -3y &  -z & = & -1 \\
         &  2y & +3z & = & 5
    \end{matrix}$

\bigskip\bigskip

\textbf{(b)} \qquad
    $\begin{matrix}
       2x_1 &  -x_2 & +3x_3 &  +x_4 & -2x_5 &  = & 3 \\
            &       &  -x_3 & +2x_4 &  -x_5 &  = & 1 \\
            &       &       &   x_4 & +4x_5 &  = & -2
    \end{matrix}$

\end{problem}
\begin{answer}
% Erase this line and put your answer here!
\end{answer}

\bigskip

\begin{problem}
The following augmented matrix represents a system of linear equations:
$$\left(\begin{array}{ccc|c}
          1 & 1 & 1 & 3\\
          2 & 3 & -1 & 4\\ 
          1 & 2 & -2 & b 
\end{array}\right)$$
For which values of the variable $b$, if any, does the system have
exactly one solution? no solution? infinitely many solutions?
\end{problem}
\begin{answer}
% Erase this line and put your answer here!
\end{answer}

\end{document}



