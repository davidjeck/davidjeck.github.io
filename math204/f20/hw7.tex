\documentclass[11pt]{article}
\usepackage[utf8]{inputenc}
\usepackage[letterpaper]{geometry}
\usepackage{amsmath}
\usepackage{amsfonts}
\usepackage{amsthm}
\usepackage{amssymb}
\usepackage{mathrsfs}
\usepackage{color}
\usepackage{graphicx}
\def\scaledgraphics#1#2{\includegraphics[width=#1]{#2}}
\setlength{\topmargin}{-0.4 true in}
\setlength{\headsep}{0 true in}
\setlength{\topskip}{0 true in}
\setlength{\textwidth}{6.5 true in}
\setlength{\oddsidemargin}{0 true in}
\setlength{\evensidemargin}{0 true in}
\setlength{\textheight}{9.2 true in}
\renewcommand{\arraystretch}{1.2}
\newcommand{\eps}{\varepsilon}
\newcommand{\R}{{\mathbb R}}
\newcommand{\N}{{\mathbb N}}
\newcommand{\Q}{{\mathbb Q}}
\newcommand{\Z}{{\mathbb Z}}
\theoremstyle{definition}
\newtheorem{problem}{Problem}
\definecolor{darkblue}{RGB}{0,0,150}
\definecolor{darkred}{RGB}{180,0,0}
\definecolor{darkgreen}{RGB}{0,120,0}
\newenvironment{answer}{\par\bigskip\bgroup\color{darkblue}}{\egroup}
\newenvironment{grading}{\par\medskip\bgroup\color{darkred}}{\egroup\par\bigskip}
\newenvironment{secondans}{\bgroup\color{darkgreen}}{\egroup}
\pagestyle{empty}


\def\rowop#1{\qquad\xrightarrow{\begin{matrix}#1\end{matrix}}\qquad}

\begin{document}


\hbox to \hsize{\bf Math 204, Fall 2020\hfil  Homework \#7}
\nointerlineskip
\vskip 2pt
\hrule height 1pt

\medskip

\centerline{\textit{This homework is due by \textbf{noon} on Tuesday, October 20}}
\centerline{\textit{\textbf{Note: Only problems 4, 5, and 6 are eligible for rewriting.}}}

\medskip


\begin{problem}
Some of the following matrix products are not defined.  For each product, you should 
either compute the product, if it is defined, or state why it is not defined.

\bigskip
\leftline{
     \hskip 1.5cm
     \textbf{a)}\quad $\begin{pmatrix} 1&2&3&4 \\ 4&3&2&1\end{pmatrix}\begin{pmatrix}-1&0\\3&-1\\-2&2\\1&0 \end{pmatrix}$
     \hskip 1.5cm
     \textbf{b)}\quad $\begin{pmatrix} 0&3&5\\ 2&1&7\\ 1&3&0\end{pmatrix}\begin{pmatrix}5&2\\3&1\end{pmatrix}$
   }
   
\bigskip
\leftline{
     \hskip 1.5cm
     \textbf{c)}\quad $\begin{pmatrix}0&1\\1&0\end{pmatrix}\begin{pmatrix}a&b\\c&d\end{pmatrix}$
     \hskip 1.5cm
     \textbf{d)}\quad $\begin{pmatrix}a&b\\c&d\end{pmatrix}\begin{pmatrix}0&1\\1&0\end{pmatrix}$
     \hskip 1.5cm
     \textbf{e)}\quad $\begin{pmatrix}3\\5\end{pmatrix}\begin{pmatrix}3&2\\1&7\end{pmatrix}$
   }

\end{problem}

\begin{answer}
% Answer here for problem number 1, 4 points
\end{answer}




\begin{problem}
Let $A=\begin{pmatrix}1\\2\\3\\4\end{pmatrix}$ and $B=\begin{pmatrix}1&2&3&4\end{pmatrix}$.
Note that $A\in M_{4\times 1}$ and $B\in M_{1\times 4}$.  Compute the matrix products $AB$ and $BA$.
\end{problem}

\begin{answer}
% Answer here for problem number 1, 2 points
\end{answer}




\begin{problem}
Find the inverse of each of the following matrices, or show that the matrix has no inverse.
For part (c), you should find the inverse by putting the augmented matrix
$$\left(\begin{array}{ccc|ccc}
 -1&3&0 & 1&0&0\\ 
 2&-1&5 & 0&1&0\\ 
 1&2&-5 & 0&0&1
\end{array}\right)$$
into reduced echelon form.

\bigskip
\leftline{
     \hskip 1.5cm
     \textbf{a)}\quad $\begin{pmatrix} 3&-2\\ 5&4 \end{pmatrix}$
     \hskip 1.5cm
     \textbf{b)}\quad $\begin{pmatrix} 6&-4\\ -3&2 \end{pmatrix}$
     \hskip 1.5cm
     \textbf{c)}\quad $\begin{pmatrix} -1&3&0 \\ 2&-1&5 \\ 1&2&-5 \end{pmatrix}$
   }

\end{problem}


\begin{answer}
% Answer here for problem number 3, 5 points
\end{answer}




\begin{problem}
We can define the derivative for all polynomials as a homomorphism $D\colon \mathscr P\to \mathscr P$.
The null space $\mathscr N(D)$ is the set of all constant polynomials.  We can form the composition
$D\circ D$, which also maps $\mathscr P$ to $\mathscr P$.  For a polynomial $q(x) 
= a_0 + a_1x+a_2x^2+\cdots+a_nx^n$, what is $D\circ D(q(x))$?  What mathematical operation does $D\circ D$
compute?  What is the null space of $D\circ D$?
\end{problem}


\begin{answer}
% Answer here for problem number 4, 3 points
\end{answer}





\begin{problem}
Suppose that $V$ and $W$ are vector spaces and that $f\colon V\to W$
is a homomorphism.  Suppose that $S\subseteq V$ and that $S$ spans $V$.
Show that the set $f(S)$ spans the range space $\mathscr R(f)$ of $f$.
(Note: $\mathscr R(f)=\{f(\vec v)\,|\,\vec v\in V\}$, and $f(S)=\{f(\vec v)\,|\,\vec v\in S\}$.)
\end{problem}

\begin{answer}
% Answer here for problem number 5, 3 points
\end{answer}




\begin{problem}
Suppose that $V$ and $W$ are vector spaces and that $f\colon V\to W$
is a homomorphism.  Assume that $h$ is one-to-one (which implies that
the null space $\mathscr N(h)$ contains only $\vec 0_V$).  And
suppose that $\langle \vec v_1,\vec v_2,\dots,\vec v_k \rangle$
is a linearly independent sequence of vectors in $V$. Show that
the sequence $\langle f(\vec v_1),f(\vec v_2),\dots,f(\vec v_k) \rangle$
is linearly independent (in $W$).
\end{problem}

\begin{answer}
% Answer here for problem number 6, 3 points
\end{answer}




\end{document}

